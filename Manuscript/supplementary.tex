\documentclass{article} 
\usepackage{graphicx}

\title{Supplementary figures}

\begin{document}

\maketitle

This supplement contains results from simulations on a metapopulation with patches arranged in a lattice structure (i.e. each habitat patch is accessible only from its four nearest neighbors).  Supplemental figures 1 and 2 are analogous to figures 1 and 2 from the main text.

\begin{figure}
\caption{The proportion of simulations resulting in endemic disease dynamics (persistence of both susceptible and infectious hosts) as a function of disease-induced mortality ($\nu$), and the probability of direct transmission ($\delta$).  Panel columns show low (0.3, a, d), medium (1.0, b, e), and high (3.2, c, f) pathogen longevities, while rows show high quality (ranging from 0.76 to 1.77, a - c) and low quality (ranging from 0.23 to 1.24, d - f) patch quality distributions.  Darker shading corresponds to more frequent endemic dynamics.  The white region above the endemic region generally corresponds to pandemic dynamics (no susceptible host populations persist), while the region below the endemic region corresponds to disease-free dynamics.}
\label{endemic}
\end{figure}

\begin{figure}
\centering
\caption{Differences between the median total effective population size of hosts on high quality patch distributions (quality between 0.76 and 1.77) and low quality patch distributions (quality between 0.23 to 1.24) as a function of disease-induced mortality ($\nu$) and the probability of direct transmission ($\delta$).  Reds indicate higher median host population sizes in high quality patch distributions, while blues indicate higher median host population sizes in low quality patch distributions.  Panel columns show low (0.3, a), medium (1.0, b), and high (3.2, c) pathogen longevities.}
\label{highvlow}
\end{figure}

\end{document}