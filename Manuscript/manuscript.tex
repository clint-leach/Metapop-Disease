%
\documentclass{svjour3} 

\usepackage{amsmath}
\usepackage{graphicx}
\usepackage{natbib}
\usepackage{tabularx}
\usepackage{lineno}
\usepackage{hyperref}
\usepackage{color}

\linenumbers

\begin{document}

\title{Environmental pathogen reservoirs and habitat heterogeneity in a metapopulation}

\author{Clinton B. Leach \and Paul C. Cross \and Colleen T. Webb}

%\institute{}
%\journalname{}


\titlerunning{Pathogen Reservoirs}

\maketitle

\section{Introduction}
\label{intro}

Many wildlife diseases are caused by pathogens that can persist, and remain infectious, for long periods of time in the environment.  Examples include chronic wasting disease (\cite{Miller2006}), anthrax (\cite{Dragon1995}), plague (\cite{Eisen2008}), and white nose syndrome (\cite{Lindner2011}), among others.  This environmental persistence creates environmental pathogen reservoirs from which hosts can become infected without direct contact with an infectious individual.  This additional transmission pathway can have important consequences for disease dynamics, with models showing that increased environmental persistence generally facilitates increased pathogen persistence and spread relative to direct transmission alone (\cite{Almberg2011}, \cite{Sharp2011}, \cite{Breban2009}). 

Since environmental transmission is spatially explicit (i.e. environmental reservoirs can only infect local residents), its role in disease dynamics may further depend on the spatial structure of the host population.  In particular, we expect that host population structure and movement will be influenced by heterogeneity in quality among the habitat patches in a metapopulation.  Indeed, the quality of a habitat patch can affect its extinction and colonization rates, as well as its contribution to the colonization of other empty patches (\cite{Moilanen1998}).  These processes in turn may then influence where environmental pathogen reservoirs get established and how they affect disease dynamics and host occupancy throughout the metapopulation.  

Specifically, we expect that high quality habitat patches, which support greater host density and traffic than lower quality habitat, might be more likely to form pathogen reservoirs.  As these reservoirs are undetectable to the host, we predict that high quality patches will continue to attract -- and infect -- susceptible immigrants, effectively creating an ecological trap (\cite{Almberg2011}).  In addition, the greater traffic through high quality patches may further facilitate pathogen spread by positioning high quality patches as metapopulation-scale superspreaders (\cite{Paull2012}).  Similarly, we expect that low quality patches, which see relatively less host traffic, will be less likely to develop pathogen reservoirs and thus may serve as refuges on which susceptible hosts can escape infection.  

Many, if not all, of the pathogens listed above affect spatially structured host populations (e.g. plague in prairie dog colonies, cite George et al 2013), and thus understanding how environmental transmission interacts with patterns of habitat quality is critical to managing disease in these systems.  In this study then, we seek to explore how a pathogen's environmental longevity (how long it can persist and remain infections in the environment), and habitat heterogeneity (the variance in the distribution of patch quality in a metapopulation) interact to influence pathogen spread and host occupancy, with a specific interest in the roles played by high and low quality habitat.  

\section{Methods}
\label{methods}

\subsection{Model Structure}

\begin{table}
\caption{State transitions and their rates for patch $i$ in a metapopulation simulation.  $S$ denotes occupied by  susceptible hosts, $I$ denotes occupied infectious hosts, and $\emptyset$ denotes unoccupied by the host (but potentially with pathogen reservoir with infection rate of $\gamma_i$).}
\begin{tabular}{lr}
State Transition & Rate \\
\hline
$S \rightarrow I$ & $\delta C_{Ii} + \gamma_i$\\
$S \rightarrow \emptyset $ & $e_{Si}$\\
$I \rightarrow \emptyset $ & $e_{Ii}$ \\
$\emptyset \rightarrow S$ & $C_{Si}$\\
$\emptyset \rightarrow I$ & $C_{Ii}$\\
\label{transitions}
\end{tabular}
\end{table}

To address the above questions, we developed a theoretical stochastic patch occupancy model (SPOM) in which each patch can be in one of three possible states: occupied by susceptible hosts (S), occupied by infectious hosts (I), and unoccupied by the host ($\emptyset$).  State transitions are governed by host colonization, extinction, and infection rates, where a susceptible population can become infectious either through direct contact with infectious immigrants, or through a local pathogen reservoir (Table~\ref{transitions}).

Building on the framework developed by \cite{Hanski2003}, the rate at which patch \emph{i}, with quality $A_i$, is colonized by individuals in state $X \in (S,I)$ is given by its connectivity:
\begin{equation}
C_{Xi}=A_i^{\xi_{im}} \sum_{i\neq j }\phi_jA_j^{\xi_{em}}e^{-D d_{ij}},
\label{connectivity}
\end{equation}
where $\phi_j$ is an indicator function that is $1$ if patch $j$ is in state $X$ and is $0$ otherwise; $\xi_{im}$ and $\xi_{em}$ control how rates of immigration and emigration, respectively, scale with patch quality; $D$ is the inverse of mean dispersal distance; and $d_{ij}$ is the distance between patches $i$ and $j$.  Essentially, $C_{Xi}$ sums the colonization effort from all patches in state $X$ to patch $i$.  

The extinction rate of patch $i$ in state $X$ is given by:
\begin{equation}
E_{Xi}=\frac{e_X}{A_i^\alpha},
\end{equation}
where $e_X$ is the extinction rate of a patch of unit quality in state $X$ (here we assume that $e_I>e_S$ to account for disease-induced mortality), and $\alpha$ controls how extinction rate scales with patch quality.  When an infectious population goes extinct, we assume that the hosts leave behind an infectious pathogen reservoir on the patch.  

Susceptible populations are infected via direct contact with infectious colonists at rate $\delta C_{Ii}$, where $\delta$ is the transmission probability.  In addition, environmental transmission can take place when a susceptible population occupies a previously infected patch, and occurs at rate:
\begin{equation}
\gamma_i=\gamma_0exp(-rt_{Ii}),
\end{equation}
where $\gamma_0$ is the initial infection rate of the pathogen reservoir, $t_{Ii}$ is the time since last infectious occupancy ($t_{Ii} = \infty$, and thus $\gamma_i = 0$, if the patch has never been occupied by infectious hosts), and $r$ is the pathogen's decay rate in the environment.


\begin{table}[h!]   
\caption{Parameters of SPOM model, their meaning, and the values assigned under different parameterizations.  Empty cells indicate same value as the default parameterization.}
\begin{tabular}{l p{5cm} rrrr}
Parameter & Interpretation &  Default & Lattice & Equal Ext. & Weak Direct Inf.\\
\hline
$\xi_{im}$ & Scaling parameter for effect of target patch quality on immigration & 0.5 & -- & -- & --\\
$\xi_{em}$ & Scaling parameter for effect of target patch quality in emigration & 0.5 & -- & -- & --\\
$D$& Inverse of mean dispersal distance & 5 & 2 & -- & --\\
$d_{ij}$ & Distance between patch $i$ and $j$ & 1 $\forall i \neq j$ & 1 $\forall i, j$ neighbors & -- & --\\
\hline
$e_S$ & Extinction rate of unit quality susceptible patch & 0.1 & -- & -- & --\\
$e_I$ & Extinction rate of unity quality infectious patch & 0.5 & -- & -- & --\\
$\alpha$ & Strength of environmental stochasticity & 1 & -- & 0 & --\\
\hline
$\delta$ & Probability of direct infection & 0.5 & -- & -- & 0.1\\
$\gamma_0$ & Initial rate of infection from reservoir patch & 0.5 & -- & -- & --
\end{tabular}
\label{params}
\end{table}

\subsection{Model Parameterization}

The model outlined above was parameterized in four different ways to explore the consequences of different assumptions (Table~\ref{params}).  The baseline model was parameterized  according to the following assumptions. (1) High quality patches maintain higher population sizes and therefore produce more colonists than low quality patches and have lower extinction rates than low quality patches ($\xi_{im}=0.5$, $\alpha=1$, \cite{Hanski2003}).  (2) High quality patches attract more colonists than low quality patches ($\xi_{em}=0.5$, \cite{Hanski2003}).  (3)  Connectivity among patches is determined entirely by quality ($d_{ij}=1$ for all $i \neq j$).  In addition to these assumptions, parameters were chosen so that without infection, approximately 0.75 of the metapopulation was occupied ($e_S = 0.1$, $D=5$), and a range of epidemiological behaviors -- disease-free, endemic disease, and pandemic disease (i.e. all host populations infectious) -- was feasible, depending on the pathogen's environmental longevity ($e_I = 0.5$, $\delta = 0.5$, $\gamma_0 = 0.5$). 

The baseline model above assumes that all patches are equally accessible from all other patches (i.e. are separated by distance 1), so to examine the effects of a more rigid spatial structure, we implemented a model with patches arranged in a square lattice such that each patch is only accessible from its four neighboring patches (Lattice, Table~\ref{params}).  In addition, we explored a model with $\alpha = 0$ so that the extinction rate was constant with patch quality, simulating the effects of high environmental stochasticity (i.e. environmental stochasticity is strong enough that the population size of a patch does not affects its extinction rate)(Equal extinction, Table~\ref{params}).  Lastly, we parameterized a model with $\delta = 0.1$ to reduce the relative importance of direct transmission and explore the disease dynamics when transmission is driven largely by the environmental reservoir (Weak direct infection, Table~\ref{params}). 

\subsection{Simulation studies}

For each of the above parameterizations, we performed two different simulation experiments to explore the combined effects of pathogen environmental longevity and habitat quality distribution on disease dynamics and metapopulation occupancy.  In the first of these, we investigated the effect of increasing habitat heterogeneity on the dynamics of pathogens with a range of environmental longevities.  For these simulations, we explored 10 values of pathogen longevity ranging from 20 to 200 time steps, with values representing the half-life of the pathogen's infectivity, and 10 values for the variance of the patch quality distribution, ranging from 0.02 to 0.2.  For each simulation (100 replicates for each of the 100 longevity-variance combinations), qualities for 100 patches were drawn from a uniform distribution with fixed mean 1 and given variance. In each case, an entirely susceptible population was simulated until it reached an approximate steady state, at which point a randomly chosen occupied patch was infected.  The state of the metapopulation was then tracked for 5000 time steps.  We recorded the proportion of patches in each state at the end of each simulation and classified the simulation as host extinction, disease-free (no infectious populations after 5000 time steps), endemic (both susceptible and infectious populations present), or pandemic (only infectious populations).  In addition, to identify the roles played by individual patches of varying quality, for every simulation, we recorded the quality of each patch, the proportion of time spent in each state (i.e. occupied by susceptible hosts, occupied by infectious hosts, and unoccupied), and the number of transitions between each state.  

The above simulations held the mean of patch quality distribution constant, and explored the effect of increasing habitat heterogeneity around that mean (i.e. by expanding the quality distribution symmetrically about 1).  To further parse out the relative influence of high and low quality patches separately, we performed a second series of simulations where we expanded the range of the uniform quality distribution to skew towards either high or low quality patches (i.e. by expanding the quality distribution asymmetrically about 1).  For these simulations, at each of the 10 longevities used above, we simulated 100 metapopulations for each of four patch quality distributions: low variance, (patch qualities ranging 0.75 to 1.25); intermediate variance, high quality (range 0.75 to 1.75); intermediate variance, low quality (range 0.25 to 1.25); and high variance (range 0.25 to 1.75). 

Continuous time stochastic simulations of the above model were implemented in the R language (\cite{R2014}) using the Gillespie algorithm (\cite{Gillespie1977}).  Code is available at \url{https://github.com/clint-leach/Metapop-Disease}.

\section{Results}
\label{results}

\textcolor{red}{Similar to above edits to methods, need to better distinguish the different analyses and what they're testing.}

In the baseline parameterization, infectious occupancy increased substantially with longevity, while susceptible and total occupancy decreased (Fig~ \ref{poutcome}).  Increasing the variance of the patch quality distribution around a mean of 1 had relatively little effect on average occupancy (Fig~\ref{poutcome}).  However, though the mean showed little change, the variance in both susceptible and infected occupancy increased with the variance of the patch quality distribution, especially at intermediate longevities.  This increase in variance is due to the greater variability in epidemiological outcome in high variance metapopulations, which generate both more pandemics and more disease die-outs than lower variance metapopulations.  The three alternative parameterizations (models with lattice structure, equal extinction rates, and weak direct infection) produced qualitatively similar results, though pathogen spread in these models was generally more limited (Figs~\ref{poutcome_lattice}, \ref{poutcome_x0}, \ref{poutcome_delta}).

\begin{figure}
\centering
\includegraphics[height=5.5cm]{metapop(full).pdf}
\caption{Mean occupancy of (a) susceptible patches, (b) infectious patches, and (c) both, as a function of the pathogen's environmental longevity (half-life of infectivity decay) and the variance of the patch quality distribution.}
\label{poutcome}
\end{figure}   

When we explored the influence of high and low quality patches separately (i.e. in simulations that expanded the patch quality distribution asymmetrically), we found that, relative to a low-variance metapopulation (patch qualities roughly between 0.75 and 1.25), shifting the distribution to include low quality patches (distribution ranging from 0.25 to 1.25) increased mean susceptible occupancy, while shifting towards high quality patches (range from 0.75 to 1.75) decreased mean susceptible occupancy, with the largest differences observed at intermediate longevities (Fig~\ref{sens}(a)).  Conversely, a shift towards low quality patches decreased infectious occupancy, while a shift towards high quality patches increased infectious occupancy (Fig~\ref{sens}(b)).  These general trends were consistent across the three other parameterizations (Figs~\ref{sens_lattice}, \ref{sens_x0}, \ref{sens_delta}).  

However, at high pathogen longevities, high quality patch distributions had a net positive effect on total occupancy relative to the other patch quality distributions (Fig~\ref{sens}(c)).  Due to their lower extinction rates and high connectivity, high quality patches were able to maintain stable occupancy despite the widespread infection facilitated by high pathogen longevity.  On the other hand, though low quality patch distributions increase total occupancy at intermediate longevities, when high pathogen longevities led to pandemic dynamics, low quality distributions were unable to support infectious occupancy, resulting in an increase in host extinction events and a decrease in mean total occupancy.  The model in which the extinction rate does not scale with quality (i.e. $\alpha = 0$) produced very similar results, though with less pronounced differences between the different quality distributions (Fig~\ref{sens_x0}(c)).  However, in simulations with weak direct infection ($\delta = 0.1$), the high quality patch distributions always produced higher occupancy than low quality patch distributions (Fig~\ref{sens_delta}(c)), while in the model with the lattice structure, low quality patch distributions actually produced higher occupancies than high quality distributions at high longevities (Fig~\ref{sens_lattice}(c)).

\begin{figure}
\centering
\includegraphics[height=7cm]{highlow(full).pdf}
\caption{The effect of the presence of high and low quality patches on (a) mean susceptible occupancy, (b) mean infectious occupancy, and (c) mean total occupancy, measured relative to  a low variance metapopulation.  Red solid lines show hiqh quality metapopulations, blue dashed show low quality, and black shows high variance.}
\label{sens}
\end{figure}

To understand the mechanisms behind these effects, we further examined within patch dynamics in the baseline model.  To compare across the full range of patch qualities, and to ensure the opportunity for both suscepible and infectious occupancy, we limit this analysis to high variance metapopulations (with qualities ranging from 0.25 to 1.75) that produced endemic dynamics.  Because of their lower extinction rate and higher recolonization rate, high quality patches supported more consistent occupancy than low quality patches (Fig~\ref{simvis}).  However, as the epidemic progressed, high quality patches increasingly supported infectious occupants, while low quality patches supported infectious hosts only briefly and infrequently (Fig~\ref{simvis}).  In addition, while high quality patches did support susceptible occupancy, they experienced a greater number of infection events (susceptible to infectious transitions) than low quality patches (Fig~\ref{infections}).  This relationship became weaker as longevity increased, due to the fact that as longevity increases and the pathogen spreads more easily, there are fewer susceptibles remaining in the metapopulation to become infectious.  A similar trend was observed across the other parameterizations, though the observed relationship between patch quality and number of infection events was shallower for the lattice and weak direct infection models (Figs~\ref{infections_lattice}, \ref{infections_x0}, \ref{infections_delta}).

\begin{figure}
\centering
\includegraphics[height = 9cm]{simvis.png}
\caption{Results from a single representative simulation with variance 0.2 and longevity 80.  The top panel shows total occupancy of susceptible (blue) and infectious (red) patches over time.  The bottom panel shows the state -- susceptible (blue), infectious (red), or unoccupied (white) -- of individual patches through time.  Patches are stacked vertically with the lowest quality ($\sim$ 0.25) at the bottom and the highest quality ($\sim$ 1.75) at the top.}
\label{simvis}
\end{figure}


\begin{figure}
\centering
\includegraphics[height=7cm]{infevents(full).pdf}
\caption{The number of infection events ($S$ to $I$ transitions) for individual patches in endemic simulations.  Light gray lines show smoothed fit for single simulations, while the thick black line shows the smoothed fit over all simulations.  The thin black lines show the smoothed fits for longevities of 40 (top) and 140 (bottom).}
\label{infections}
\end{figure}


\section{Discussion}
\label{discussion} 

The distribution of habitat quality in a metapopulation can have substantial impacts on the dynamics of environmentally transmitted pathogens and the resulting patterns of host occupancy.  These impacts are driven largely by the contrasting effects of low and high quality habitat on pathogen transmission, and their relative balance in the metapopulation. \textcolor{red}{Need more here? Possibly discussing some of the background colonization, extinction processes that are modified by patch quality, and how generally those interact with environmental transmission.}  

Since a patch's connectivity is determined by its quality (Eqn~\ref{connectivity}), high quality patches function as highly connected hubs in the metapopulation.  Specifically, high quality patches both attract more colonists from other patches and produce more colonists that spread to other patches.  These two properties have important implications for pathogen spread in the metapopulation. Attracting colonists helps position high quality patches as the ecological "traps" hypothesized above, wherein susceptible hosts repeatedly colonize high quality patches and subsequently become infected (Fig~\ref{infections}).  This infection pressure on high quality patches results from both transmission from the environmental reservoir and from direct transmission from infectious immigrants attracted to the high quality patch, reflecting results from the contact network literature that show that highly connected nodes have higher infection risk (\cite{Christley2005}, \cite{Keeling2005}).  

This trap effect, coupled with the relatively low extinction rate on high quality patches helps to create a stable platform from which the pathogen can spread through the rest of the metapopulation.  Indeed, high quality patches are effectively metapopulation scale superspreaders (\cite{Lloyd-Smith2005}, \cite{Paull2012}).  Once the pathogen infects high quality patches, the larger number of host colonists produced by these patches allow it to spread to the rest of the metapopulation relatively easily.  This process helps to maintain infectious occupancy throughout the metapopulation, which in turn feeds back on high quality patches to ensure a steady stream of infectious colonists that help maintain the environmental reservoir and the trap effect.  Thus, through these two interacting mechanisms -- the trap and superspreader effects -- the presence of high quality patches serves to significantly increase pathogen spread and infectious occupancy in the metapopulation (Fig~\ref{sens}).  

The importance of high quality habitat in maintaining pathogen spread suggests that high quality habitat may be a good target for control efforts, e.g. quarantine or vaccination.  This echoes the insights of \cite{Hess1996} who simulated the spread of a directly transmitted pathogen through a metapopulation and found that implementing a quarantine on centralized patches reduced pathogen spread.  This insight is even more important for environmentally transmitted pathogens.

In contrast to high quality habitat, low quality patches help to limit pathogen spread and increase susceptible occupancy (Fig~\ref{sens}). Even though individual low quality patches, due to their high extinction rates and low colonization rates, are unable to support consistent occupancy (Fig~\ref{simvis}), the presence of low quality patches increases overall susceptible occupancy relative to a lower variance metapopulation (Fig~\ref{sens}).  This phenomenon represents the other side of the hub role played by high quality patches in that low quality patches are relatively weakly connected to the metapopulation and are therefore infrequently colonized by infectious individuals.  Moreover, the instability of low quality patches (i.e. the high extinction rate), together with the increased mortality of infectious populations, means that infectious populations do not persist long on low quality patches.  Because low quality patches are so infrequently colonized, at low to intermediate longevities, the environmental reservoir left behind by these infectious populations generally decays before it has the opportunity to infect new susceptible colonists.  As a result, low quality patches represent a dead-end for the pathogen, reducing the number of patches from which it can spread, and thereby reducing infectious occupancy (Fig~\ref{sens}).  \textcolor{red}{Maybe also note that natural mortality on low quality patches is high enough that there's not enough time for susceptibles to to get infected.}

Though this model provides evidence that high quality patches can serve as traps and superspreaders, and low quality patches can serve as susceptible refuges, the effects of these processes on total metapopulation occupancy depend on the strength and structure of the two transmission pathways.  \textcolor{red}{Need more here?}  In particular, these findings suggest the presence of a trade-off between between processes that facilitate metapopulation stability at the expense of disease spread, and processes that hinder disease spread at the expense of metapopulation stability.  The position of these trade-offs, and their relative importance depend largely on the strength and structure of the different avenues of pathogen transmission.  Broadly, there are three different scenarios: very weak pathogen spread, intermediate/endemic pathogen spread, and easy/pandemic pathogen spread.  When pathogen spread is limited or slow (e.g. due to low longevity, or weak direct transmission), high quality patches are a net benefit to the metapopulation, as their connectivity and stability facilitate greater occupancy with little additional pathogen spread.  

However, in cases where the pathogen spreads more easily (i.e. at an endemic level), the addition of high quality patches functions similarly to increasing movement in the model developed by \cite{Hess1996}, increasing pathogen spread and reducing overall occupancy (Fig~\ref{sens}c).  In these cases, where pathogen longevity is intermediate, or pathogen spread is limited by metapopulation structure, rather than serving as sources in the metapopulation, high quality patches become sinks.  More importantly, they become ecological traps, which can be even more detrimental to the overall population (\cite{Kristan2003}).  From a management perspective, it is critical to understand this shift, as previous theoretical work suggests that management should focus on maintaining patches where conditions are most favorable (\cite{Strasser2010}).  Generally, this means that managers should focus on maintaining high quality habitat, but Strasser et al. (2010) show that stochastic disturbance (e.g. disease-induced mortality) can lead to cases where focusing on low quality patches is more effective in increasing population growth rate.  Indeed, in these situations, we see that low quality patch distributions, by hindering the spread of the pathogen and providing refuges for susceptibles, produce higher occupancy than high quality distributions.  

Also discuss \cite{Sharp2011}, who show that management actions that increase carrying capacity can increase disease spread and destabilize population

As pathogen longevity increases, the trade-off shifts and the addition of high quality patches is a net benefit to overall occupancy, reflecting the findings of \cite{Gog2002}, who find that under background infection from an alternative host, increasing movement rarely reduces occupancy, despite facilitating pathogen spread.  \cite{Park2012} further develops this work by noting that the magnitude of the detrimental effects of increased moevement on occupancy depends on the relative strength of different transmission pathways.  Specifically, he finds that as environmental transmission becomes stronger (e.g. higher longevity) relative to direct transmission, then total occupancy increases monotonically with movement, i.e. movement is a net benefit to the metapopulation.  These studies, coupled with the results presented here, suggest that when there is a persistent source of infection throughout the metapopulation (e.g. when the pathogen is long-lived in the environment), factors that facilitate metapopulation stability (i.e. increased movement, high quality patches) are a greater benefit than factors that inhibit pathogen spread (i.e. decreased movement, low quality patches). 

\textcolor{red}{What does this model offer beyond the literature cited?  WRT Hess 1996, the extension to explicit habitat patches offers insight into the roles of specific patches; even though connectivity is an important driving factor in this model, it also allows us to consider the role of other processes (e.g. extinction, explicity pathogen transmission in space).}

\textcolor{red}{As with the notes above, emphasize not just how this work fits with other literature, but also what it adds.  WRT these other metapop disease modeling papers, may not want to equate increased patch quality with increased movement so easily and instead delve more into what including patch quality offers us.  "This refelcts the the results of blah in that ..., but offers the additional insight that ..."}

These previous studies have focused movement as the mechanism behind this trade-off.  Shifting the patch quality distribution towards high quality patches in this model is in many ways analogous to increasing movement in the models of Hess, Gog, and Park.  However, by habitat heterogeneity and metapopulation structure, the model presented here allows us to drill down below the metapopulation level and investigate the role of individual patches.  In the Park model, movement is homogeneous, and increasing movement increases pathogen exposure equally across all patches, whereas in the model here, movement is heterogeneous and directional.  In our model, we're actually holding per capita movement rates constant, and changing the structure and direction of that movement.  



Spatial structure and disease spread represent two major factors influencing the dynamics of wildlife populations.  As we have seen, in cases where the pathogen is able to persist and remain infectious in the environment, these two factors can interact in complex and interesting ways.  Failing to consider these interactions and their effect on the roles played by high and low quality habitat risks overlooking the importance of low-quality habitat for the persistence of wildlife populations and the importance of high-quality habitat for the persistence of wildlife disease.

\textcolor{red}{Again, use this last paragraph not just to summarize the results, but to emphasize the novelty of these insights.}

\clearpage

\section{Appendix}

\subsection{Sensitivity Analyses}

\subsubsection{Lattice structure}

\begin{figure}[h!]
\centering
\includegraphics[height=5.5cm]{metapop(lattice).pdf}
\caption{Percent of simulations resulting in (a) endemic disease (both susceptibles and infectiouss persist) and (b) pandemic disease (no susceptibles persist), as a function of the pathogen's environmental longevity (half-life of infectivity decay) and the variance of the patch quality distribution.}
\label{poutcome_lattice}
\end{figure}   

\begin{figure}
\centering
\includegraphics[height=7cm]{highlow(lattice).pdf}
\caption{The effect of the presence of high and low quality patches on (a) mean susceptible occupancy, (b) mean infectious occupancy, and (c) mean total occupancy, measured relative to  a low variance metapopulation.  Red solid lines show hiqh quality metapopulations, blue dashed show low quality, and black shows high variance.}
\label{sens_lattice}
\end{figure}

\begin{figure}
\centering
\includegraphics[height=7cm]{infevents(lattice).pdf}
\caption{The number of infection events ($S$ to $I$ transitions) for individual patches in endemic simulations.  Light gray lines show smoothed fit for single simulations, while the thick black line shows the smoothed fit over all simulations.  The thin black lines show the smoothed fits for longevities of 40 (bottom) and 140 (top).}
\label{infections_lattice}
\end{figure}

\clearpage

\subsubsection{Extinction scaling, $\alpha = 0$}

\begin{figure}[h!]
\centering
\includegraphics[height=5.5cm]{metapop(x0).pdf}
\caption{Percent of simulations resulting in (a) endemic disease (both susceptibles and infectiouss persist) and (b) pandemic disease (no susceptibles persist), as a function of the pathogen's environmental longevity (half-life of infectivity decay) and the variance of the patch quality distribution.}
\label{poutcome_x0}
\end{figure}   

\begin{figure}
\centering
\includegraphics[height=7cm]{highlow(x0).pdf}
\caption{The effect of the presence of high and low quality patches on (a) mean susceptible occupancy, (b) mean infectious occupancy, and (c) mean total occupancy, measured relative to  a low variance metapopulation.  Red solid lines show hiqh quality metapopulations, blue dashed show low quality, and black shows high variance.}
\label{sens_x0}
\end{figure}

\begin{figure}
\centering
\includegraphics[height=7cm]{infevents(x0).pdf}
\caption{The number of infection events ($S$ to $I$ transitions) for individual patches in endemic simulations.  Light gray lines show smoothed fit for single simulations, while the thick black line shows the smoothed fit over all simulations.  The thin black lines show the smoothed fits for longevities of 40 (top) and 140 (bottom).}
\label{infections_x0}
\end{figure}

\clearpage

\subsubsection{Weak direct transmission, $\delta = 0.1$}

\begin{figure}[h!]
\centering
\includegraphics[height=5.5cm]{metapop(delta).pdf}
\caption{Percent of simulations resulting in (a) endemic disease (both susceptibles and infectiouss persist) and (b) pandemic disease (no susceptibles persist), as a function of the pathogen's environmental longevity (half-life of infectivity decay) and the variance of the patch quality distribution.}
\label{poutcome_delta}
\end{figure}   

\begin{figure}
\centering
\includegraphics[height=7cm]{highlow(delta).pdf}
\caption{The effect of the presence of high and low quality patches on (a) mean susceptible occupancy, (b) mean infectious occupancy, and (c) mean total occupancy, measured relative to  a low variance metapopulation.  Red solid lines show hiqh quality metapopulations, blue dashed show low quality, and black shows high variance.}
\label{sens_delta}
\end{figure}

\begin{figure}
\centering
\includegraphics[height=7cm]{infevents(delta).pdf}
\caption{The number of infection events ($S$ to $I$ transitions) for individual patches in endemic simulations.  Light gray lines show smoothed fit for single simulations, while the thick black line shows the smoothed fit over all simulations.  The thin black lines show the smoothed fits for longevities of 60 (bottom) and 120 (top).}
\label{infections_delta}
\end{figure}

\clearpage

\bibliographystyle{spbasic}     
\bibliography{metapop}   % name your BibTeX data base





\end{document}
