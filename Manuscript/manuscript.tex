%
\documentclass{svjour3} 

\usepackage{amsmath}
\usepackage{graphicx}
\usepackage{natbib}
\usepackage{tabularx}
\usepackage{lineno}
\usepackage{hyperref}

\linenumbers

\begin{document}

\title{Environmental pathogen reservoirs and habitat heterogeneity in a metapopulation}

\author{Clint B. Leach \and Paul C. Cross \and Colleen T. Webb}

%\institute{}
%\journalname{}

\date{November 26, 2013}

\titlerunning{Pathogen Reservoirs}

\maketitle

\section{Introduction}
\label{intro}

Many wildlife diseases are caused by pathogens that can persist, and remain infectious, for long periods of time in the environment.  Examples include chronic wasting disease (\cite{Miller2006}), anthrax (\cite{Dragon1995}), plague (\cite{Eisen2008}), and white nose syndrome (\cite{Lindner2011}), among others.  This environmental persistence creates environmental pathogen reservoirs from which hosts can become infected without direct contact with an infected individual.  This additional transmission pathway can have important consequences for disease dynamics, especially in spatially structured populations (\cite{Park2012}).

Patterns and rates of host movement also play an important role in pathogen spread through a metapopulation.  As shown by \cite{Hess1996}, for pathogens spread by direct contact alone, increased host movment enhances pathogen spread through a metapopulation.  Further studies by \cite{Gog2002} and \cite{McCallum2002} explore the role of additional transmission pathways by including background infection from an alternate host, while \cite{Park2012} includes transmission from an environmental reservoir.  These studies show that the the effect of host movement on occupancy (and pathogen spread) depends on the relative strength of the different transmission pathways and highlight the potential for patterns of host movement and pathways of pathogen transmission to interact and influence disease dynamics.   

However, the above studies assume that all patches in the metapopulation are identical, and consider movement only at the metapopulation scale.  Since environmental transmission is spatially explicit (i.e. environmental reservoirs can only infect residents of the local patch), its role in disease dynamics may depend on the attributes of the specific habitat patches in which reservoirs become established.  In particular, the quality of a habitat patch can affect its extinction and colonization probabilities, as well as its contribution to the colonization of other empty patches (\cite{Moilanen1998}).  As an example, in a study of the Glanville fritillary butterfly metapopulation, \cite{Moilanen1998} found that increased patch quality, as measured by the density of flowering plants, decreased emigration and increased immigration.  Thus, patterns of host movement, and consequently the role of environmental pathogen transmission, are determined at least in part by patch quality.  

Many of the pathogens listed above affect spatially structured host populations (e.g. plague in prairie dog colonies), and thus understanding the consquences of environmental pathogen persistence and habitat quality for pathogen spread is critical to managing disease in these systems.  In this study, we ask how pathogen longevity (how long the pathogen can persist and remain infections in the environment), and habitat heterogeneity (the variance in the distribution of patch quality across the metapopulation) interact to influence pathogen spread through the metapopulation.  Within a metapopulation, we ask what specific role low and high quality patches play in controlling pathogen spread.  

We expect pathogen spread to increase with both environmental longevity and with habitat heterogeneity.  Increased environmental longevity should provide greater opportunities for transmission, and thus should favor pathogen spread.  The role of habitat heterogeneity is likely to be more complex, emerging from the behavior of high and low quality patchs in heterogeneous metapopulations.  Specifically, we expect that high quality habitat patches, which support greater host traffic than lower quality habitat, might be more likely to form pathogen reservoirs.  As these reservoirs are undetectable to the host, we predict that high quality patches will continue to attract -- and infect -- susceptible immigrants, effectively creating an ecological trap (citations, examples?).  In addition, the greater traffic through high quality patches may further facilitate pathogen spread by positioning high quality patches as metapopulation-scale superspreaders (citation).  Similarly, we predict that low quality patches, which see relatively less host traffic, will be less likely to develop pathogen reservoirs and and thus may serve as refuges for susceptible hosts to escape infection.  

To address the above questions and evaluate our hypotheses, we build on the metapopulation framework developed by \cite{Hanski1994} to develop and explore a theoretical stochastic patch occupancy model that incorporates pathogen transmission from environmental reservoirs and habitat heterogeneity.   

\section{Methods}
\label{methods}

\subsection{Model Structure}
To address the above questions, we developed a theoretical stochastic patch occupancy model (SPOM) in which each patch can be in one of four possible states: occupied by susceptibles (S), occupied by infecteds (I), pathogen reservoir without host population (Z), and unoccupied (by both pathogen and host) ($\oslash$).  State transitions are governed by host colonization, extinction, and infection rates, where a susceptible population can become infected either through direct contact with infected immigrants, or through a local pathogen reservoir.  

Building on the framework developed by \cite{Hanski1994}, the rate at which patch \emph{i}, with quality $A_i$, is colonized by individuals in state $X \in (S,I)$ is given by its connectivity:
\begin{equation}
C_{Xi}=A_i^{\xi_{im}} \sum_{i\neq j }\phi_jA_j^{\xi_{em}}e^{-D d_{ij}},
\end{equation}
where $\phi_j$ is an indicator function that is $1$ if patch $j$ is in state $X$ and is $0$ otherwise; $\xi_{im}$ and $\xi_{em}$ control how rates of immigration and emigration, respectively, scale with patch quality; $D$ is the inverse of mean dispersal distance; and $d_{ij}$ is the distance between patches $i$ and $j$.  Essentially, $C_{Xi}$ sums the colonization effort from all patches in state $X$ to patch $i$.  

The extinction rate of patch $i$ in state $X$ is given by:
\begin{equation}
E_{Xi}=\frac{e_X}{A_i^\alpha},
\end{equation}
where $e_X$ is the extinction rate of a patch of unit quality in state $X$ (here we assume that $e_I>e_S$).  When an infected patch goes extinct, we assume that the infected hosts leave behind an infectious pathogen reservoir.  

Susceptible populations are infected via direct contact with infected colonists at rate $\delta C_{Ii}$, where $\delta$ is the transmission probability.  Environmental transmission occurs when a susceptible population occupies a reservoir patch, and occurs at rate:
\begin{equation}
\gamma=\gamma_0exp(-rt_I),
\end{equation}
where $\gamma_0$ is the initial infection rate of the pathogen reservoir, $t_I$ is the time since last infected occupancy, and $r$ is the pathogen's decay rate in the environment.


\begin{table}[h!]
\label{parameters}      
\caption{Parameters of SPOM model and their meanings.}
\begin{tabular}{l p{8.5cm} r}
Parameter & Interpretation &  Default Value\\
\hline
$\xi_{im}$ & Scaling parameter for effect of target patch quality on immigration & 0.5\\
$\xi_{em}$ & Scaling parameter for effect of target patch quality in emigration & 0.5\\
$D$& Inverse of mean dispersal distance & 5\\
$d_{ij}$ & Distance between patch $i$ and $j$ & 1 for all $i,j$\\
\hline
$e_S$ & Extinction rate of unit quality susceptible patch & 0.1\\
$e_I$ & Extinction rate of unity quality infected patch & 0.5\\
$\alpha$ & Strength of environmental stochasticity & 1\\
\hline
$\delta$ & Probability of direct infection & 0.5\\
$\gamma_0$ & Initial rate of infection from reservoir patch & 0.5
\end{tabular}
\end{table}

\subsection{Model Implementation and Analysis}

The model outlined above was parameterized (Table 1) according to the following assumptions. (1) High quality patches maintain higher population sizes and therefore produce more colonists than low quality patches and have lower extinction rates than low quality patches ($\xi_{im}=0.5$, $\alpha=1$, \cite{Hanski2003}).  (2) High quality patches attract more colonists than low quality patches ($\xi_{em}=0.5$, \cite{Hanski2003}).  (3)  Connectivity is determined entirely by quality ($d_{ij}=1$).  In addition to these assumptions, parameters were chosen so that without infection, approximately 0.75 of the metapopulation was occupied ($e_S = 0.1$, $D=5$), and a range of epidemiological behaviors were feasible ($e_I = 0.5$, $\delta = 0.5$, $\gamma_0 = 0.5$). 

With the above parameters fixed, we explored the effects of pathogen environmental persistence and habitat heterogeneity by varying pathogen longevity (10 values from 20 to 200 time steps, with values representing the half-life of the pathogen's infectivity), and the variance of the patch quality distribution (10 values ranging from 0.02 to 0.2).  For each simulation (100 replicates for each of the 100 longevity-variance combinations), qualities for 100 patches were drawn from a uniform distribution with mean 1 and given variance. In each case, an entirely susceptible population was simulated until it reached a steady state, at which point a randomly chosen occupied patch was infected.  The state of the metapopulation was then tracked for 5000 time steps.  To address the metapopulation-level questions above, we recorded the proportion of patches in each state at the end of each simulation and classified the simulation as host extinction, disease-free (no infected populations after 5000 time steps), endemic (both susceptible and infected populations present), or pandemic (only infected populations).  In addition, to address patch-level questions, for every simulation, we recorded the quality of each patch and the proportion of time spent in each of the four states.  

In the above simulations, the mean of the patch quality distribution was held constant, and thus the range was expanded symmetrically about one as the variance was increased.  To further parse out the relative roles of high and low quality patches, we performed a series of simulations where we expanded the range of the quality distribution non-symmetrically to include high and low quality patches separately.  For these simulations, we fixed longevity at an intermediate level (100) and ran 100 replicates for each of four patch quality distributions: low varaince, (range 0.75 to 1.25); intermediate variance, high quality (range 0.75 to 1.75); intermediate variance, low quality (range 0.25 to 1.25); and high variance (range 0.25 to 1.75).

Continuous time stochastic simulations of the above model were implemented in the R language (citation) using the Gillespie algorithm (\cite{Gillespie1977}).  Code is available at \url{https://github.com/clint-leach/Metapop-Disease}.

\section{Results}
\label{results}

The pathogen persisted in 6954 of the 10000 simulations, spreading through the entire susceptible host population (i.e. a pandemic) in 3734 simulations.  The probability of pathogen persistence increased substantially with pathogen longevity, though patch quality variance had relatively little effect (Figure 1a).  Within cases where the pathogen persisted however, the probability of a pandemic increased with both pathogen longevity and patch quality variance (Figure 1b).

\begin{figure}
\label{poutcome}
\centering
\includegraphics[height=5.5cm]{metapop(full).pdf}
\caption{Logistic regression surface showing the probability of (a) pathogen persistence, and (b) pandemic spread given persistence, as a function of pathogen environmental longevity (half-life of infectivity decay) and patch quality variance.}
\end{figure}   

To separate the effects of high and low quality patches, we explore how patch occupancy changes when high and low quality patches are included in the patch quality distribution separately.  In low variance metapopulations (where patch quality ranges from 0.75 to 1.25, corresponding to a variance of 0.02), the proportions of patches occupied by susceptible populations and infected populations are approximately equal, with occupancies of 0.26 and 0.24 respectively.  However, when the patch quality distribution is expanded to include high quality patches (range from 0.75 to 1.75), susceptible occupancy drops to nearly zero, while infected occupancy increases to approximately half of the metapopulation (Figure 2).  Conversely, when the patch quality distribution is expanded to include low quality patches (range from 0.25 to 1.25), susceptible occupancy increases to nearly half, while infected occupancy declines (Figure 2).  When both high and low quality patches are included (range from 0.25 to 1.75, corresponding to a variance of 0.2), their effects approximately cancel, reducing both susceptible and infected occupancy slightly (Figure 2).  Though overall occupancy changes little from low to high variance metapopulations, the proportion of simulations that are pandemics still increases (Figure 1), an effect that can be attributed largely to high quality patches.  However, in high variance metapopulations, the number of disease-free simulations also increases, resulting in relatively little change in mean occupancy.  \emph{Replace these results with outcome-level results for consistency.}

\begin{figure}
\label{sens}
\centering
\includegraphics[height=5cm]{highlow(full).pdf}
\caption{The effect of the presence of high and low quality patches on (a) mean infected occupancy and (b) mean susceptible occupancy. Fix to represent the correct experimental design and results.}
\end{figure}

To understand the mechanism behind this effect, we look to within patch dynamics, namely the proportion of time spent by individual patches in each of the four possible states.  We limit our analysis to high variance endemic simulations (where the pathogen persists but does not infect the entire susceptible population).  This suggests that high quality patches are more frequently occupied by both infected and susceptible populations (Figure 3).  However, while the proportion of time occupied by infecteds increases steadily with patch quality (Figure 3b), the proportion of time occupied by susceptibles plateaus (Figure 3a).  Moreover, high quality patches tend to experience a greater number of infection events than low quality patches (Figure 4).

\begin{figure}
\label{qualityraw}
\centering
\includegraphics[height=7cm]{qualityraw(full).pdf}
\caption{Scatter cloud showing the proportion of time spent occupied by infecteds and susceptibles for individual patches in endemic simulations. Darker shades of blue represent higher point density. The red line represents the loess curve fit to the data.}
\end{figure}

\begin{figure}
\label{infections}
\centering
\includegraphics[height=7cm]{infevents(full).pdf}
\caption{Scatter cloud showing the number of infection events ($S$ to $I$ transitions) for individual patches in endemic simulations. Darker shades of blue represent higher point density. The red line represents the loess curve fit to the data.}
\end{figure}

\section{Discussion}
\label{discussion} 

The positive relationship between pathogen persistence/spread and environmental longevity demonstrated in the theoretical model above is well documented in models of empirical systems.  \cite{Almberg2011} find that, in plausible simulations of chronic wasting disease in deer, increasing prion survival in the environment increases the resulting force of infection and peak prevalence, with the  importance of reservoir transmission increasing through time.  Similarly, Breban et al 2009 show that environmental transmission of avian influenza virus allows it to persist in situations where direct transmission alone is insufficient.  \emph{Other examples.}

When longevity is sufficient for the pathogen to persist, we find that habitat heterogeneity controls how and where it spreads.  In particular, increased habitat quality variance facilitates pandemic dynamics, increasing the likelihood that the pathogen will spread through the entire host population.  The mechanisms underlying this behavior can best be understood by drawing parallels between our metapopulation model and models of pathogen spread through a contact network.  In this framework, our metapopulation becomes a network with patches as nodes connected through immigration (with the strength of the connection measured by connectivity).  Many studies have shown that networks with high degree variance (where a node's degree is roughly equivalent to its connectivity in our metapopulation model) are easily invaded by pathogens, which are then able to spread much more rapidly than in less heterogeneous networks (\cite{Pastor-Satorras2001}).

This increase in pathogen spread in high variance metapopulations is largely attributable to the presence of high quality patches (Figure 2).  Since metapopulation connectivity is determined entirely by patch quality (recall $d_{ij}=1$ for all $i, j$), high quality patches function as highly connected hubs.  Borrowing vocabulary from network theory, high quality patches are effectively high degree nodes (i.e. many connections).  Specifically, the assumptions of the model mean that high quality patches have both high in-degree (attract colonists from other patches) and a high out-degree (produce colonists that spread to other patches).  These two properties have important implications for pathogen spread in the metapopulation. The high in-degree helps position high quality patches as the ecological "traps" hypothesized above, wherein susceptibles repeatedly colonize high quality patches and subsequently become infected (Figure 4).  This infection pressure is the result of both the environmental reservoir and direct infection from infected immigrants attracted to the high quality patch.  This reflects the results from the contact network literature that show that highly connected (i.e. high degree) nodes have higher infection risk than less connected nodes (\cite{Christley2005}, \cite{Keeling2005}).  This trap effect, coupled with the relatively low extinction rate on high quality patches helps to create a stable platform from which the pathogen can spread through the rest of the metapopulation.  Indeed, high quality patches are effectively metapopulation scale superspreaders (\cite{Lloyd-Smith2005}).  Once the pathogen infects high quality patches, it is able to spread to the rest of the metapopulation relatively easily.  This process helps to maintain infected patches throughout the metapopulation, which in turn feed back on high quality patches to ensure a steady stream of infected colonists that maintain the trap effect.  Thus, through these two interacting mechanisms -- the trap and superspreader effects -- the presence of high quality patches in high variance metapopulations serves to significantly increase pathogen spread.  

Despite the dominating effects of high quality patches, low quality patches still play an important role in some cases by providing refuges on which susceptibles can escape infection. Even though individual low quality patches are unable to support consistent occupancy (susceptible or otherwise), at the metapopulation level, the presence of low quality patches increases overall susceptible occupancy, suggesting that in aggregate, low quality patches can provide susceptible refuges.  This phenomenon represents the other side of the hub role played by high quality patches in that low quality patches are relatively weakly connected to the metapopulation and are therefore infrequently colonized by infected individuals.  This is consistent with the result that low degree nodes tend to be the last infected in models of pathogen spread on contact networks (\cite{Barthelemy2004}).  Moreover, the instability of low quality patches (i.e. the high extinction rate), together with the increased mortality of infected populations, means that infected individuals do not persist long on low quality patches.   However, it is important to caveat this refuge effect by noting that it appears to be swamped in the presence of too many high quality patches, as seen by the fact that the likelihood of pandemic increases with patch quality variance despite the fact that high variance metapopulations have equal proportions of high and low quality patches.  This suggests that once enough high quality patches are added, the force of infection in the system becomes large enough to overwhelm the weak connectivity of low quality patches.  

In metapopulations without disease, high quality patches generally serve as sources, while low quality patches are sinks.  The presence of a pathogen then considerably alters, and even inverts, these patterns.  Not only do high quality patches become sinks in the presence of a pathogen with an environmental reservoir, they become ecological traps, which can be even more detrimental to the overall population (\cite{Kristan2003}).  From a management perspective, it is critical to understand this shift, as previous theoretical work suggests that management should focus on patches where conditions are most favorable (\cite{Strasser2010}).  Generally, this means that maintaining high quality habitat should form the focus of managers, but Strasser et al. (2010) show that stochastic disturbance (e.g. disease) can lead to cases where focusing on low quality patches is more effective in increasing population growth rate.

Habitat fragmentation and disease represent two major threats to wildlife populations.  As we have seen, these two factors can interact in complex and interesting ways, especially when the pathogen is able to persist in the environment.  Failing to consider these interactions and their effect on the roles played by high and low quality habitat overlooks the importance of low-quality habitat for the persistence of wildlife populations and the importance of high-quality habitat for the persistence of wildlife disease.




\bibliographystyle{spbasic}     
\bibliography{metapop}   % name your BibTeX data base





\end{document}
