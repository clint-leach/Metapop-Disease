\documentclass{article} 

\usepackage{amsmath}
\usepackage{graphicx}
\usepackage{tabularx}
\usepackage{lineno}
\usepackage{hyperref}
\usepackage{color}
\usepackage{authblk}
\usepackage[T1]{fontenc}
\usepackage{pslatex}

\linenumbers

\begin{document}

\title{When environmentally persistent pathogens transform good habitat into ecological traps}

\author[1]{Clinton B. Leach$^*$}
\author[1]{Colleen T. Webb}
\affil[1]{Department of Biology, Graduate Degree Program in Ecology, Colorado State University, Fort Collins, CO, USA}
\author[2]{Paul C. Cross}
\affil[2]{U.S. Geological Survey, Northern Rocky Mountain Science Center, Bozeman, Montana, USA}
\affil[*]{Corresponding author: clint.leach@colostate.edu}


\maketitle


\begin{quote}
This draft manuscript is distributed solely for purposes of scientific peer review.  Its content is deliberative and pre decisional, so it must not be disclosed or released by reviewers.  Because the manuscript has not yet been approved for publication by the U.S. Geological Survey (USGS), it does not represent any official USGS finding or policy.
\end{quote}

\newpage

\begin{abstract} 
The benefits of high quality habitat for host persistence may be modulated by the presence of an environmentally persistent pathogen. In some cases, the presence of environmental pathogen reservoirs on high quality habitat may lead to the creation of ecological traps, wherein host individuals preferentially colonize high quality habitat, but are then exposed to increased infection risk and disease-induced mortality.  We explored this possibility through the development of a stochastic patch occupancy model, where we varied the pathogen's virulence, transmission rate, and environmental persistence as well as the distribution of habitat quality in the host metapopulation.  This model suggests that for pathogens with intermediate levels of spread, high quality habitat can serve as an ecological trap, and can be detrimental to host persistence relative to low quality habitat.  This inversion of the relative roles of high and low quality habitat highlights the importance of considering the interaction between spatial structure and pathogen transmission when managing wildlife populations exposed to an environmentally persistent pathogen.
\end{abstract}

\subsubsection*{Keywords:}
disease; metapopulation; habitat quality; environmental transmission; ecological trap

\section{Introduction}
\label{intro}

Many prominent and problematic wildlife diseases are caused by pathogens that can persist, and remain infectious, for long periods of time in the environment.  Examples include chronic wasting disease (\cite{Miller2006}), anthrax (\emph{Bacillus anthracis}, \cite{Dragon1995}), plague (\emph{Yersinia pestis}, \cite{Eisen2008}), and white nose syndrome (\emph{Pseudogymnoascus destructans}, \cite{Reynolds2015}), among others.  This environmental persistence creates environmental pathogen reservoirs from which susceptible hosts can become infected without direct contact with an infectious individual.  This additional transmission pathway can have important consequences for disease dynamics, with models showing that increased environmental longevity generally facilitates increased pathogen persistence and spread relative to direct transmission alone (\cite{Almberg2011}, \cite{Sharp2011}, \cite{Breban2009}). 

Since environmental transmission is spatially explicit (i.e. environmental reservoirs can only infect local residents), its role in disease dynamics may further depend on the spatial structure of the host population.  In particular, we expect that host population structure and movement will be influenced by heterogeneity in quality among the habitat patches in a metapopulation.  Indeed, the quality of a habitat patch can affect its extinction and colonization rates, as well as its contribution to the colonization of other empty patches (\cite{Moilanen1998}).  These processes will influence how a pathogen spreads, where environmental pathogen reservoirs get established, and the resulting effects on host occupancy and population size.  

We expect that high quality habitat patches, which support greater host density and traffic than lower quality habitat, might be more likely to form pathogen reservoirs and support consistently infectious populations.  As these reservoirs are undetectable to the host, we predict that high quality patches will continue to attract -- and infect -- susceptible immigrants, effectively creating an ecological trap (\cite{Robertson2006}, \cite{Almberg2011}).  In addition, the greater traffic through high quality patches may further facilitate pathogen spread by positioning high quality patches as metapopulation-scale superspreaders (\cite{Paull2012}).  Similarly, we expect that low quality patches will be less likely to develop pathogen reservoirs and may serve as refuges where susceptible hosts can escape infection.
The combination of these effects may then lead to the creation of metapopulation-scale traps, where the inclusion of otherwise high quality habitat in the metapopulation actually reduces host population size relative to low quality habitat.

Many, if not all, of the pathogens listed above affect spatially structured host populations (e.g. plague in prairie dog colonies, \cite{George2013}), and thus understanding how pathogen transmission interacts with patterns of habitat quality is critical to managing disease in these systems.  In this study, we explore how pathogen spread interacts with the distribution of habitat quality to influence host occupancy.  We specifically focus on the relative importance of high and low quality habitats to host persistence in the presence of pathogens with different levels of environmental longevity. 

\section{Methods}
\label{methods}

\subsection{Stochastic patch occupancy model}


To address the above questions, we developed a stochastic patch occupancy model (SPOM) in which each patch can be in one of three possible states: occupied by susceptible hosts (S), occupied by infectious hosts (I), and unoccupied by the host ($\emptyset$).  State transitions are governed by host colonization, extinction, and transmission rates, where a susceptible population can become infectious either through direct contact with infectious immigrants, or through a local environmental pathogen reservoir (Table~\ref{transitions}).

We assume that the host population size supported by a patch is proportional to its quality ($A_i$), and that a proportion, $\nu$, of hosts survive when a population is infected (i.e. the size of infectious populations is reduced by a factor of $\nu$).  Building on the framework developed by \cite{Hanski2003} then, the rates at which patch \emph{i}, with quality (or equivalently population size) $A_i$, is colonized by susceptible and infectious individuals, respectively, is given by its connectivity:
\begin{align}
C_{Si}(t) &= A_i^{\xi_{im}} \sum_{i\neq j }\phi_j(S, t)A_j^{\xi_{em}}\exp(-D d_{ij}),\\
C_{Ii}(t) &= A_i^{\xi_{im}} \sum_{i\neq j }\phi_j(I, t)(\nu A_j) ^{\xi_{em}}\exp(-D d_{ij}),
\label{connectivity}
\end{align}
where $\phi_j(X, t)$ is an indicator function that is $1$ if patch $j$ is in state $X$ at time $t$ and is $0$ otherwise; $\xi_{im}$ and $\xi_{em}$ control how rates of immigration and emigration, respectively, scale with patch quality/population size; $D$ is the inverse of the host's mean dispersal distance; and $d_{ij}$ is the effective distance between patches $i$ and $j$ .  Essentially, $C_{Xi}$ sums the colonization effort (or propagule pressure) from all patches in state $X$ to focal patch $i$.  

The state-dependent extinction rates of populations on patch $i$ are then given by:
\begin{align}
E_{Si} & =\frac{\mu}{A_i^\alpha},\\
E_{Ii} & = \frac{\mu}{(\nu A_i) ^ \alpha}
\end{align}
where $\mu$ is the extinction rate of a patch of unit quality, and $\alpha$ controls how extinction rate scales with population size.  When an infectious population goes extinct, we assume that the hosts leave behind an infectious pathogen reservoir on the patch.  

Susceptible populations are infected via direct contact with infectious colonists at rate $\delta C_{Ii}$, where $\delta$ is the direct transmission probability.  In addition, environmental transmission can take place when a susceptible population occupies a previously infected patch, and occurs at rate:
\begin{equation}
\gamma(\tau_i)=\gamma_0exp(-r\tau_{i}),
\end{equation}
where $\gamma_0$ is the initial transmission rate of the pathogen reservoir, $\tau_{i}$ is the time since last infectious occupancy ($\tau_{i} = \infty$, and thus $\gamma(\tau_i) = 0$, if the patch has never been occupied by infectious hosts), and $r$ is the pathogen's decay rate in the environment (determined by the pathogen's longevity, defined as its half-life in the environment).

\subsection{Model parameterization}

The model was parameterized according to the following assumptions (Table~\ref{params}). 
(1) The rate of emigration from a patch increases with the local population size, while the extinction rate of a patch decreases with populations size ($\xi_{im}=0.5$, $\alpha=1$, \cite{Hanski2003}).  
(2) Migrants preferentially select high quality habitat, that is, the rate at which a patch is colonized scales with quality ($\xi_{em}=0.5$, \cite{Hanski2003}, though we relax this below).  
(3)  Connectivity among patches is determined entirely by quality ($d_{ij}=1$ for all $i \neq j$).  
In addition to these assumptions, we chose parameters so that without infection, approximately 0.75 of the patches were occupied ($\mu = 0.1$, $D=5$).  We varied infectious survival ($\nu$), probability of direct transmission ($\delta$), and environmental longevity to explore the dynamics of a range of possible pathogens.

The parameterization above assumes that all patches are equally accessible from all other patches (i.e. are separated by effective distance 1).  To examine the effects of a more rigid spatial structure, we also implemented a model with patches arranged in a square lattice such that each patch is only accessible from its four neighbouring patches (in which case, average migration distance was adjusted to $D = 2$ to maintain roughly equivalent occupancy).  

\subsection{Simulation studies}

To explore the relative influence of low and high quality habitat on host and pathogen dynamics, we simulated the spread of a range of pathogens in metapopulations with patch quality distributions shifted towards either high or low quality habitat.
These two distributions were chosen to have equal variance and an overlapping core of patches around unit quality, with either the minimum or maximum shifted to simulate regions with more, or less, high quality habitat (yielding distributions with qualities ranging from 0.23 to 1.24 and from 0.76 to 1.77).  
We then implemented a full-factorial set of simulations on each quality distribution in which infectious survival ($\nu$) and probability of direct transmission ($\delta$) were each varied over 10 values (ranging from 0.1 to 1, and 0 to 0.9, respectively).  In addition, pathogen longevity was varied over 3 values (ranging from 0.3 to 3 times the expected residence time of a healthy population on a unit quality patch, i.e., 10 units of time).

For each simulation -- 100 replicates for each combination of disease parameters ($\nu$, $\delta$, and longevity), habitat quality distribution (high or low), and landscape structure (fully connected or lattice), qualities for 100 patches were drawn from a uniform distribution on the given habitat quality range (above). In each case, an entirely susceptible population was simulated until it reached an approximate steady state, at which point a randomly chosen occupied patch was infected.  The state of the metapopulation was then tracked for 5000 time steps.  We recorded the proportion of patches in each state at the end of each simulation and also computed the mean total population size of susceptible and infectious hosts over the last 500 time steps of the simulation (summing the local population sizes for all occupied patches, assuming a local population size of $A_i$ for patches that are occupied and susceptible and $\nu A_i$ for occupied infectious patches).    

To more carefully explore the role of habitat quality in within-patch dynamics, we performed an additional simulation experiment on metapopulations with patch qualities evenly spaced over the full range explored above (from 0.2 to 1.8).
In these simulations, we varied longevity over the values used above and explored the consequences of host preference for high quality habitat (and the corresponding potential for ecological traps) by choosing two different values for $\xi_{im}$ (with $\xi_{im} = 0.5$ indicating preference for high quality habitat, and  $\xi_{im} = 0$ indicating that migrants select all patches with equal probability).  
To isolate the effects of habitat preference and pathogen longevity, we fixed the direct transmission probability ($\delta = 0.3$), infectious survival ($\nu = 0.2$), and landscape structure (fully connected).
We performed 100 replicate simulations for each set of model parameters (three values of pathogen longevity and two values of $\xi_{im}$) and recorded the proportion of time each patch spent occupied by susceptible and infectious hosts, and the final total host population size.

Continuous time stochastic simulations of the above model were implemented in the R language (\cite{R2014}) using the Gillespie algorithm (\cite{Gillespie1977}).  Code is available at \\
\url{https://github.com/clint-leach/Metapop-Disease}.

\section{Results}
\label{results}

A wide range of disease dynamics were observed over the range of parameters and habitat quality distributions explored.  
Epidemic outcome was influenced most strongly by infectious survival, with high infectious survival (i.e. low disease-induced mortality) facilitating widespread infection (in which no susceptible host populations persist), and very low infectious survival (i.e. high disease-induced mortality) facilitating disease-free dynamics or intermediate spread (Fig~\ref{endemic}). 
Pathogens with higher environmental longevities were generally able to invade and spread more easily, even with low infectious survival (Fig~\ref{endemic}(c, f)).  
The distribution of habitat quality in the metapopulation also played an important role in determining disease dynamics.  Increasing overall habitat quality facilitated widespread infection for a larger range of parameters (Fig~\ref{endemic}, a-c vs. d-f).
Metapopulations with a lattice structure exhibited qualitatively similar patterns, but with generally more limited pathogen spread and larger regions of susceptible maintenance (Supplement).

For a wide range of parameter values, high quality patch distributions had a net positive effect on total host population size relative to low quality distributions despite generally facilitating pathogen spread (red shaded regions in Fig~\ref{highvlow}).  
In the case of pathogens that were unable to invade due to low infectious survival and weak transmission, high quality habitat distributions supported larger total host populations than low quality distributions due to the greater stability and connectivity afforded by high quality habitat (bottom left corner of the panels in Fig~\ref{highvlow}). 
Similarly, when high infectious survival and strong transmission (from either environmental or direct transmission) allowed a pathogen to spread widely (i.e. at a pandemic level) in both habitat quality distributions, high quality distributions again supported larger total host populations than low quality distributions (top portion of panels in Fig~\ref{highvlow}).

However, for pathogens falling between these extremes (e.g. with intermediate infectious survival), low quality habitat was a net benefit to the host, supporting larger total population sizes than high quality habitat (blue regions of Fig ~\ref{highvlow}).
The spread of these pathogens was strongly influenced by the habitat quality distribution, with low quality habitat limiting disease spread and better maintaining susceptible occupancy relative to high quality habitat (Fig~\ref{popsizes}, Supplemental).
In these cases, high quality habitat served as a sort of metapopulation-level trap, that is, otherwise beneficial habitat that is detrimental to the overall host population in the presence of disease.
As pathogen longevity increased, the range of the parameter space where we observed this metapopulation-level trap became smaller, apparent only for the most virulent pathogens.
Again, metapopulations with a lattice structure displayed similar patterns (Supplement).

To better understand the patch-level mechanisms behind these effects, we further examined within patch infection dynamics for parameters in this intermediate range ($\nu = 0.2$ and $\delta = 0.3$).  
Because of their lower extinction rate and higher recolonization rate, high quality patches were more consistently occupied than low quality patches (Fig~\ref{simvis}, Supplement).  
However, because of this stability, disease risk, measured by the proportion of time a patch spent occupied by infectious hosts, increased with patch quality (Supplement).
Indeed, as an epidemic progressed, high quality patches increasingly supported infectious occupants, while low quality patches supported infectious hosts only briefly and infrequently (Fig~\ref{simvis}).  
Thus, from the perspective of a migrant individual, colonizing high quality habitat posed increased disease risk, with the preferential selection of high quality habitat (i.e. $\xi_{im} = 0.5$) increasing this disease risk further (Supplement).
Moreover, assuming preferential selection of high quality habitat had dramatic consequences for the stability of the host metapopulation when the pathogen is long-lived in the environment, increasing the probability of extinction substantially compared to random habitat selection (0.29 vs. 0.03).

\section{Discussion}
\label{discussion} 

This work demonstrates that the habitat quality in a metapopulation can have substantial impacts on the dynamics of a variety of environmentally transmitted pathogens and the resulting size and stability of the host population.
In some cases, the presence of high quality habitat can enhance the spread and impact of environmentally persistent pathogens.  
In addition, these patches with many resources may become metapopulation-scale ecological traps that are normally beneficial and attractive to hosts, but become a net drain on the metapopulation due to the impacts of disease. 
This suggests that the presence of a disease can lead to a trade-off between otherwise high quality habitat that facilitates metapopulation stability at the expense of increased disease spread, and low quality habitat that hinders disease spread at the expense of metapopulation stability.

This expands on previous studies of disease spread in a homogeneous metapopulation that focused on movement rate as the driver of this trade-off, with increased movement both improving metapopulation stability and increasing disease spread (\cite{Hess1996}, \cite{Gog2002}, \cite{Park2012}).  
Shifting the patch quality distribution towards high quality habitat in our model is conceptually similar to increasing movement in these models.
And, as in these models, the net consequences of this trade-off for the host metapopulation depend on the characteristics of the disease.
Broadly, we can divide the observed dynamics into three different scenarios: limited, intermediate, and widespread infection (possibly leading to host extinction). 
High quality habitats are a net benefit to the total host population when pathogen spread is either limited or is high enough that both high and low quality patches are affected by disease. 
In these cases, the connectivity and stability of high quality patches facilitate larger total population sizes with little additional pathogen spread.

There is an intermediate scenario, however, where low quality habitats are more likely to remain uninfected and as a result boost the total population size while high quality patches that are heavily infected tend to reduce the total population size. 
In this scenario, the ability of pathogens to invade and spread is influenced substantially by the colonization and extinction rates of high and low quality habitat, and the contrasting roles that these habitat types play in the metapopulation.

\subsubsection*{Role of high quality habitat}

Because high quality habitat supports larger local host population sizes, these populations are less susceptible to environmental stochasticity and thus experience less frequent extinction events.
Due to this additional stability, high quality habitat is better able to both maintain a consistent pool of susceptible hosts to which infection can spread, and support infectious host populations under disease-induced mortality. 
As a result, the probability of infection (i.e. proportion of time occupied by infectious hosts) increases with patch quality (Supplement).
From the perspective of a migrant susceptible host then, selection of high quality habitat leads to increased disease risk relative to other habitats, either because the patch is likely already occupied by infectious hosts, or because the local susceptible hosts are likely to soon become infected (either through the presence of an environmental reservoir or direct contact).

Thus, if high quality habitat is preferred by migrants (i.e. $\xi_{em} > 0$), it forms an ecological trap at the individual scale, that is, preferentially selected habitat that reduces individual fitness relative to other habitats (\cite{Robertson2006}).
Moreover, the preferential selection of high quality habitat further exacerbates the increase in disease risk with habitat quality (Supplement) by increasing the risk of direct transmission from infectious migrants also selecting high quality habitat.
This echoes results from the contact network literature that show that highly connected nodes have higher infection risk (\cite{Christley2005}, \cite{Keeling2005}).

The relatively low extinction rate on high quality patches, and the resulting increased probability of infection, helps to create a relatively stable platform from which the pathogen can spread through the rest of the metapopulation.
Indeed, for pathogens with spread in this intermediate range, high quality patches are effectively metapopulation scale superspreaders (\cite{Lloyd-Smith2005}, \cite{Paull2012}).  
This helps to maintain infectious occupancy throughout the metapopulation, which, when coupled with preference for high quality habitat, feeds back on high quality patches to ensure a steady stream of new infectious colonists. 
Through these two interacting processes -- the ecological trap created by the preference of individuals for high quality habitat, and the superspreader role emerging from the increased connectivity and stability -- the presence of high quality patches serves to significantly increase pathogen spread, even for more virulent pathogens (Fig~\ref{endemic}, \ref{popsizes}).
The consequences of widespread disease-induced mortality can outweigh the positive consequences of increased metapopulation connectivity and stability, leading to the creation of a metapopulation-scale trap and a net decrease in total host population size on high quality habitat distributions (Fig~\ref{highvlow}, \ref{popsizes}).  

\subsubsection*{Role of low quality habitat}

In contrast to high quality habitat, low quality patches help to limit pathogen spread and increase susceptible occupancy in the metapopulation (Fig~\ref{endemic}, \ref{popsizes}). 
Even though individual low quality patches, due to their high extinction and low colonization rates, are only able to support transient occupancy (Fig~\ref{simvis}, Supplement), the presence of low quality habitat facilitates more widespread susceptible persistence for a greater range of pathogens (Fig~\ref{endemic}).  
This phenomenon represents the other side of the role played by high quality patches in that low quality patches are relatively unstable (i.e. small local population sizes and high extinction rates) and thus do not support infectious populations for long.
Since they are infrequently colonized (when high quality habitat is preferentially selected, $\xi_{em} > 0$), the environmental reservoir left behind by these infectious populations likely decays before it has the opportunity to infect new susceptible colonists. 
As a result, low quality patches effectively represent a dead-end for the pathogen, reducing the number of patches from which it can spread, and allowing susceptible hosts to more easily persist and avoid infection throughout the metapopulation.
The opportunity for pathogen reservoir decay provided by the ephemeral occupancy of low quality habitat echoes the 'migratory escape' hypothesis (\cite{Loehle1995}), which suggests that annual migration allows hosts to vacate contaminated sites and return once they are clean.
Taken with this literature, our results highlight the importance of transient/temporary habitat in mitigating host exposure to environmentally transmitted pathogens.

\subsubsection*{Consequences for management}

From a management perspective, it is critical to understand this shift in the effect of habitat quality in order to avoid falling into the metapopulation trap created by high quality habitat and its interaction with pathogen spread.
Previous theoretical work suggests that management should focus on maintaining patches where conditions are most favourable for the host (i.e. high quality habitat), but Strasser et al. (\cite{Strasser2010}) show that stochastic disturbance (e.g. disease-induced mortality) can lead to cases where focusing on low quality habitat is more effective in increasing population growth rate.
Ovaskainen and Hanski (\cite{Ovaskainen2003a}) similarly demonstrate that correlated extinctions reduce the contributions of well-connected habitat to metapopulation persistence and increase the contributions of more isolated habitat.
Our model adds to this literature by demonstrating the potential for environmentally persistent pathogens to provide the kind of disturbance under which low quality habitat is beneficial, and thus should be conserved.

\subsection*{Conclusions}  

This partitioning of pathogen dynamics into three regions based on whether high quality habitat is a net benefit or detriment to the host metapopulation echoes the work of \cite{Hess1996}, \cite{Gog2002}, \cite{Park2012}.
In particular, for a given pathogen, Hess (\cite{Hess1996}) similarly identified three regions of disease dynamics which determined whether increased movement had a positive or negative effect on host occupancy.  
In cases where the host movement rate was either low enough that the pathogen was unable to invade or high enough to cause a widespread infection, Hess found that further increasing movement increased host occupancy, while at intermediate levels that resulted in moderate disease spread, increasing movement decreased host occupancy.
These three scenarios, distinguished in \cite{Hess1996} by the host movement rate, map to the three scenarios we outline here based on disease parameters (infectious survival and direct and environmental transmission).
The conceptual similarity of these results suggests more generally that in metapopulations facing intermediate pathogen spread, factors that would normally increase host stability and connectivity in the absence of disease (e.g. increased movement, high quality habitat) can actually decrease total host population size as a result of increased pathogen spread.

However, we find that as the environmental longevity of the pathogen increases, the region in which we observe this reversal becomes smaller, that is, high quality habitat is a metapopulation-level trap for a smaller range of potential pathogens (Fig~\ref{highvlow}, blue region shrinks from panels a. to c.).  
Environmentally long-lived pathogens are able to spread widely under a larger range of conditions, largely overwhelming the ability of low quality habitat to provide refuges for susceptible hosts. 
These dynamics reflect the findings of Gog \emph{et al.} (\cite{Gog2002}), who found that under strong background infection from an alternative host, increasing movement always increased total occupancy, despite facilitating pathogen spread.  
Park (\cite{Park2012}) expanded on this by adding environmental transmission to the mix and found that in cases where background and environmental transmission were strong relative to direct transmission, increasing movement was again a net benefit to the host metapopulation.  
These studies, coupled with the results presented here, suggest that when there is a persistent source of infection throughout the metapopulation (e.g. from long-lived environmental reservoirs), factors that facilitate metapopulation stability (e.g. increased movement, high quality habitat) are generally a greater benefit than factors that inhibit pathogen spread (e.g. decreased movement, low quality habitat). 

The three scenarios identified here provide a rough framework that we can use to identify the potential risk of metapopulation-level traps in wildlife populations currently affected by an environmentally persistent pathogen.
For instance, high quality habitat seems unlikely to form metapopulation traps for bat populations affected by white nose syndrome due to the widespread infection both within and between hibernaculum and the scarcity of host refuges (\cite{Langwig2014}, \cite{ORegan2015}).
Similarly, from these results, we would not expect the spread of chronic wasting disease to turn high quality habitat into traps for mule deer metapopulations, due again to widespread infection (albeit at low prevalence) across population units (\cite{Conner2004}) and the relative importance of local-scale transmission within winter ranges (\cite{Farnsworth2006}).
However, we might expect to see high quality traps in the plague-prairie dog system, where the infection is widespread and disease-induced mortality is high, but uninfected colonies are still able to persist (\cite{Stapp2004}).
Indeed, studies have suggested that large colonies are more likely to become infected (\cite{Snall2008}), and thus high quality habitat capable of supporting large populations may help facilitate plague persistence and spread.
Thus, by considering the characteristics of a host-pathogen system and their effect on the roles played by high and low quality habitat, we can begin to diagnose where empirical systems fall relative to these three scenarios and the corresponding consequences for management.
As we have seen, failing to do so risks overlooking the importance of low-quality habitat for the persistence of wildlife populations and the importance of high-quality habitat for the persistence of wildlife disease.

\subsection*{Data accessibility}
Simulation source code and generated data are available at \\  \url{https://github.com/clint-leach/Metapop-Disease}.

\subsection*{Competing interests}
We have no competing interests.

\subsection*{Author's contributions}
CBL participated in the design of the study, wrote the code and carried out the simulations, and drafted the manuscript; CTW and PCC conceived of the study, participated in the design of the study, and helped draft the manuscript.  All authors gave final approval for publication.

\subsection*{Acknowledgements}

CBL would like to thank members of the Webb Lab at Colorado State University, past and present, for several rounds of useful feedback.  

\subsection*{Funding}

This work was supported by USGS Cooperative Agreement 101485, and NSF Graduate Research Fellowship, No. DGE-1321845.
\\
\\
Any mention of trade, product, or firm names is for descriptive purposes only and does not imply endorsement by the U.S. Government.

\clearpage

\bibliographystyle{vancouver}     
\bibliography{metapop}   % name your BibTeX data base

\clearpage

\begin{figure}
\includegraphics[angle=270, scale=0.5]{figure/figure_1.eps}
\caption{Median total susceptible population size as a function of infectious survival ($\nu$), and the probability of direct transmission ($\delta$).  Panel columns show low (0.3, a, d), medium (1.0, b, e), and high (3.2, c, f) pathogen longevities, while rows show low quality (ranging from 0.23 to 1.24, a - c) and high quality (ranging from 0.76 to 1.77, d - f) patch quality distributions.  Darker shading corresponds to larger population sizes, while white indicates that no susceptible host populations persist (i.e. all host populations become infected).}
\label{endemic}
\end{figure}

\begin{figure}
\includegraphics[angle=270, scale=0.5]{figure/figure_2.eps}
\centering
\caption{Differences between the median total effective population size of hosts on low quality patch distributions (quality between 0.23 to 1.24) and high quality patch distributions (quality between 0.76 and 1.77) as a function of infectious survival ($\nu$) and the probability of direct transmission ($\delta$).  Reds indicate that high quality habitat distributions support higher median host population sizes than low quality habitat, primarily corresponding to pathogens with either very limited or very extensive spread. Conversely, blues indicate higher median host population sizes on low quality habitat distributions, corresponding to pathogens with intermediate levels of spread.  Panel columns show low (0.3, a), medium (1.0, b), and high (3.2, c) pathogen longevities.}
\label{highvlow}
\end{figure}

\begin{figure}
\includegraphics[angle=270, scale=0.5]{figure/figure_3.eps}
\centering
\caption{Boxplots showing the range, over 100 replicate simulations, of susceptible (a), infectious (b), and total (c) host population sizes for high quality (red) and low quality (blue) habitat distributions.  Infectious survival ($\nu$) and direct transmission rate ($\delta$) are fixed at values of 0.2 and 0.3, respectively.  Coloured points show medians, while vertical lines indicate the inner-quartile ranges, with horizontal lines indicating the minimum and maximum.  At low longevities, this pathogen is unable to invade and high quality habitat supports larger populations of susceptible hosts.  However, low quality habitat is better able to maintain these susceptible hosts as the pathogen's longevity increases, leading to larger total population sizes.}
\label{popsizes}
\end{figure}

\begin{figure}
\includegraphics[angle=270, scale=0.5]{figure/figure_4.eps}
\centering
\caption{Results from a single representative simulation with habitat quality uniformly distributed between 0.23 and 1.77, and a pathogen with an infectious survival of $\nu = 0.2$, a direct transmission rate of $\delta = 0.3$, and an envioronmental longevity equal to the expected residence time of a susceptible population on a unit quality patch (10 units of time here).  The top panel shows total proportion of patches occupied by susceptible (blue) and infectious (red) hosts over time.  The bottom panel shows the state -- susceptible (blue), infectious (red), or unoccupied (white) -- of individual patches through time.  Patches are stacked vertically with the lowest quality at the bottom and the highest quality at the top.  Note the overall more consistent occupancy on high quality habitat and the only very brief spurts of infectious occupancy on low quality habitat.}
\label{simvis}
\end{figure}

\clearpage

\begin{table}
\caption{State transitions and their rates for patch $i$ in a metapopulation simulation.  $S$ denotes occupied by  susceptible hosts, $I$ denotes occupied infectious hosts, and $\emptyset$ denotes unoccupied by the host.  All patches are characterized by an environmental transmission rate, $\gamma(\tau_i)$, where $\tau_i$ is the time since the patch was last occupied by infectious individuals.}
\begin{tabular}{llr}
State Transition & Process &  Rate \\
\hline
$S \rightarrow I$ & Transmission (from contact or reservoir) & $\delta C_{Ii} + \gamma(\tau_i)$\\
$S \rightarrow \emptyset $ & Extinction (of susceptible) & $E_{Si}$\\
$I \rightarrow \emptyset $ & Extinction (of infectious) &  $E_{Ii}$ \\
$\emptyset \rightarrow S$ & Colonization (by susceptibles) & $C_{Si}$\\
$\emptyset \rightarrow I$ & Colonization (by infectious) & $C_{Ii}$\\
\label{transitions}
\end{tabular}
\end{table}

\begin{table}[h!]   
\caption{Parameters of the SPOM model, their meaning, and the values, or range of values, assigned.}
\begin{tabular}{p{1cm} p{6cm} r}
Param. & Interpretation &  Value(s) \\
\hline
$\xi_{im}$ & Effect of patch quality on immigration (preference for high quality habitat) & 0, 0.5 \\
$\xi_{em}$ & Effect of population size on emigration & 0.5  \\
$D$& Inverse of mean dispersal distance & 2 (lattice), 5 (full) \\
$d_{ij}$ & Distance between patch $i$ and $j$ & \parbox[t]{4cm}{\raggedleft $1 \forall i, j$ neighbors (lattice)\\ $1 \forall  i\neq j$ (full)}\\
\hline
$\mu$ & Extinction rate of unit quality patch & 0.1 \\
$\nu$ & Infectious survival & 0.1 - 1 \\
$\alpha$ & Strength of environmental stochasticity & 1 \\
\hline
$\delta$ & Probability of direct transmission & 0 - 0.9 \\
$\gamma_0$ & Initial rate of transmission from reservoir patch & 0.5
\end{tabular}
\label{params}
\end{table}

\end{document}
