%
\documentclass{svjour3} 

\usepackage{amsmath}
\usepackage{graphicx}
\usepackage{natbib}
\usepackage{tabularx}
\usepackage{lineno}
\usepackage{hyperref}

\linenumbers

\begin{document}

\title{Environmental pathogen reservoirs and habitat heterogeneity in a metapopulation}

\author{Clinton B. Leach \and Paul C. Cross \and Colleen T. Webb}

%\institute{}
%\journalname{}


\titlerunning{Pathogen Reservoirs}

\maketitle

\section{Introduction}
\label{intro}

Many wildlife diseases are caused by pathogens that can persist, and remain infectious, for long periods of time in the environment.  Examples include chronic wasting disease (\cite{Miller2006}), anthrax (\cite{Dragon1995}), plague (\cite{Eisen2008}), and white nose syndrome (\cite{Lindner2011}), among others.  This environmental persistence creates environmental pathogen reservoirs from which hosts can become infected without direct contact with an infected individual.  


\cite{Almberg2011} find in models of CWD that increasing prior survival increases pathogen spread (using a number of measures) relative to a model that only accounts for direct transmission; highlights importance of accounting for environmental transmission.

\cite{Breban2009} find that including environmental transmission better explains AIV in North American waterfowl than direct transmission only, and that it allows for persistence in cases where direct transmission alone would not.

\cite{Clancy2005} shows that environmental transmission facilitates pathogen persistence compared to models with direct transmission only.

\cite{Fuller2012} quantify persistence time baculovirus in gypsy moths and suggest that quantifying persistence times in empirical systems and including those estimates in models offers greater understanding of disease spread.  (should read this more carefully)

\cite{Sharp2011} model CWD and find that as longevity increases, critical carrying capacities (that trigger disease persistence first, and then stable limit cycles and recurrent epidemics) are reduced, and thus populations are more susceptible to disease outbreaks.  Also suggest that management efforts that increase carrying capacity (e.g. high quality habitat) can facilitate disease persistence and possibly destabilize the population.



Since environmental transmission is spatially explicit (i.e. environmental reservoirs can only infect residents of the local patch), its role in disease dynamics may depend on the attributes of the specific habitat patches in which reservoirs become established.  In particular, the quality of a habitat patch can affect its extinction and colonization probabilities, as well as its contribution to the colonization of other empty patches (\cite{Moilanen1998}).  Thus, we expect that patterns of host movement, and consequently the effect of environmental pathogen transmission, are determined at least in part by patch quality.  

Many of the pathogens listed above affect spatially structured host populations (e.g. plague in prairie dog colonies), and thus understanding how environmental pathogen persistence interacts with habitat quality is critical to managing disease in these systems.  In this study, we ask how pathogen longevity (how long the pathogen can persist and remain infections in the environment), and habitat heterogeneity (the variance in the distribution of patch quality across the metapopulation) interact to influence pathogen spread and metapopulation occupancy.  Within a metapopulation, we ask what specific role low and high quality patches play in controlling pathogen spread.  

We expect pathogen spread to increase with both environmental longevity and with habitat heterogeneity.  Increased environmental longevity should provide greater opportunities for transmission, and thus should favor pathogen spread.  The role of habitat heterogeneity is likely to be more complex, emerging from the behavior of high and low quality patchs in heterogeneous metapopulations.  Specifically, we expect that high quality habitat patches, which support greater host traffic than lower quality habitat, might be more likely to form pathogen reservoirs.  As these reservoirs are undetectable to the host, we predict that high quality patches will continue to attract -- and infect -- susceptible immigrants, effectively creating an ecological trap (\cite{Almberg2011}, citations, examples?).  In addition, the greater traffic through high quality patches may further facilitate pathogen spread by positioning high quality patches as metapopulation-scale superspreaders (citation).  Similarly, we predict that low quality patches, which see relatively less host traffic, will be less likely to develop pathogen reservoirs and and thus may serve as refuges for susceptible hosts to escape infection.  

To address the above questions and evaluate our hypotheses, we build on the metapopulation framework developed by \cite{Hanski1994} to develop and explore a theoretical stochastic patch occupancy model that incorporates pathogen transmission from environmental reservoirs and habitat heterogeneity.   

\section{Methods}
\label{methods}

\subsection{Model Structure}
To address the above questions, we developed a theoretical stochastic patch occupancy model (SPOM) in which each patch can be in one of four possible states: occupied by susceptibles (S), occupied by infecteds (I), pathogen reservoir without host population (Z), and unoccupied (by both pathogen and host) ($\oslash$).  State transitions are governed by host colonization, extinction, and infection rates, where a susceptible population can become infected either through direct contact with infected immigrants, or through a local pathogen reservoir.  

Building on the framework developed by \cite{Hanski1994}, the rate at which patch \emph{i}, with quality $A_i$, is colonized by individuals in state $X \in (S,I)$ is given by its connectivity:
\begin{equation}
C_{Xi}=A_i^{\xi_{im}} \sum_{i\neq j }\phi_jA_j^{\xi_{em}}e^{-D d_{ij}},
\end{equation}
where $\phi_j$ is an indicator function that is $1$ if patch $j$ is in state $X$ and is $0$ otherwise; $\xi_{im}$ and $\xi_{em}$ control how rates of immigration and emigration, respectively, scale with patch quality; $D$ is the inverse of mean dispersal distance; and $d_{ij}$ is the distance between patches $i$ and $j$.  Essentially, $C_{Xi}$ sums the colonization effort from all patches in state $X$ to patch $i$.  

The extinction rate of patch $i$ in state $X$ is given by:
\begin{equation}
E_{Xi}=\frac{e_X}{A_i^\alpha},
\end{equation}
where $e_X$ is the extinction rate of a patch of unit quality in state $X$ (here we assume that $e_I>e_S$).  When an infected patch goes extinct, we assume that the infected hosts leave behind an infectious pathogen reservoir.  

Susceptible populations are infected via direct contact with infected colonists at rate $\delta C_{Ii}$, where $\delta$ is the transmission probability.  Environmental transmission occurs when a susceptible population occupies a reservoir patch, and occurs at rate:
\begin{equation}
\gamma=\gamma_0exp(-rt_I),
\end{equation}
where $\gamma_0$ is the initial infection rate of the pathogen reservoir, $t_I$ is the time since last infected occupancy, and $r$ is the pathogen's decay rate in the environment.


\begin{table}[h!]
\label{parameters}      
\caption{Parameters of SPOM model and their meanings.}
\begin{tabular}{l p{8.5cm} r}
Parameter & Interpretation &  Default Value\\
\hline
$\xi_{im}$ & Scaling parameter for effect of target patch quality on immigration & 0.5\\
$\xi_{em}$ & Scaling parameter for effect of target patch quality in emigration & 0.5\\
$D$& Inverse of mean dispersal distance & 5\\
$d_{ij}$ & Distance between patch $i$ and $j$ & 1 for all $i,j$\\
\hline
$e_S$ & Extinction rate of unit quality susceptible patch & 0.1\\
$e_I$ & Extinction rate of unity quality infected patch & 0.5\\
$\alpha$ & Strength of environmental stochasticity & 1\\
\hline
$\delta$ & Probability of direct infection & 0.5\\
$\gamma_0$ & Initial rate of infection from reservoir patch & 0.5
\end{tabular}
\end{table}

\subsection{Model Implementation and Analysis}

The model outlined above was parameterized (Table 1) according to the following assumptions. (1) High quality patches maintain higher population sizes and therefore produce more colonists than low quality patches and have lower extinction rates than low quality patches ($\xi_{im}=0.5$, $\alpha=1$, \cite{Hanski2003}).  (2) High quality patches attract more colonists than low quality patches ($\xi_{em}=0.5$, \cite{Hanski2003}).  (3)  Connectivity is determined entirely by quality ($d_{ij}=1$).  In addition to these assumptions, parameters were chosen so that without infection, approximately 0.75 of the metapopulation was occupied ($e_S = 0.1$, $D=5$), and a range of epidemiological behaviors were feasible ($e_I = 0.5$, $\delta = 0.5$, $\gamma_0 = 0.5$). 

With the above parameters fixed, we explored the effects of pathogen environmental persistence and habitat heterogeneity by varying pathogen longevity (10 values from 20 to 200 time steps, with values representing the half-life of the pathogen's infectivity), and the variance of the patch quality distribution (10 values ranging from 0.02 to 0.2).  For each simulation (100 replicates for each of the 100 longevity-variance combinations), qualities for 100 patches were drawn from a uniform distribution with mean 1 and given variance. In each case, an entirely susceptible population was simulated until it reached a steady state, at which point a randomly chosen occupied patch was infected.  The state of the metapopulation was then tracked for 5000 time steps.  To address the metapopulation-level questions above, we recorded the proportion of patches in each state at the end of each simulation and classified the simulation as host extinction, disease-free (no infected populations after 5000 time steps), endemic (both susceptible and infected populations present), or pandemic (only infected populations).  In addition, to address patch-level questions, for every simulation, we recorded the quality of each patch and the proportion of time spent in each of the four states.  

In the above simulations, the mean of the patch quality distribution was held constant, and thus the range was expanded symmetrically about one as the variance was increased.  To further parse out the relative roles of high and low quality patches, we performed a series of simulations where we expanded the range of the quality distribution non-symmetrically to include high and low quality patches separately.  For these simulations, we fixed longevity at an intermediate level (100) and ran 100 replicates for each of four patch quality distributions: low varaince, (range 0.75 to 1.25); intermediate variance, high quality (range 0.75 to 1.75); intermediate variance, low quality (range 0.25 to 1.25); and high variance (range 0.25 to 1.75).

Continuous time stochastic simulations of the above model were implemented in the R language (citation) using the Gillespie algorithm (\cite{Gillespie1977}).  Code is available at \url{https://github.com/clint-leach/Metapop-Disease}.

\section{Results}
\label{results}

The pathogen persisted in 7087 of the 10000 simulations, spreading through the entire susceptible host population (i.e. a pandemic) in 3721 simulations.  Pathogen persistence increased substantially with longevity, with intermediate longevities favoring endemic disease, and high longevities facilitating pandemic dynamics (Fig~ \ref{poutcome}).  The effect of longevity, however, depended on the variance of the patch quality distribution, with low variance metapopulations favoring endemic dynamics for a wider range of longevities (Fig~ \ref{poutcome}a), and high variance metapopulations lowering the longevity required for pandemic dynamics (Fig~ \ref{poutcome}b).

\begin{figure}
\centering
\includegraphics[height=5.5cm]{metapop(full).pdf}
\caption{Percent of simulations resulting in (a) endemic disease (both susceptibles and infecteds persist) and (b) pandemic disease (no susceptibles persist), as a function of the pathogen's environmental longevity (half-life of infectivity decay) and the variance of the patch quality distribution.}
\label{poutcome}
\end{figure}   

To separate the influence of high and low quality patches, we explored how susceptible and infected patch occupancy changes when high and low quality patches are included separately in the patch quality distribution.  Relative to a low-variance metapopulation (patch qualities roughly between 0.75 and 1.25), shifting the distribution toward low quality patches (distribution ranging from 0.25 to 1.25) increased mean susceptible occupancy, while shifting towards high quality patches (range from 0.75 to 1.75) decreased mean susceptible occupancy, with the largest effects observed at intermediate longevities (Fig~\ref{sens}(a)).  Conversely, a shift towards low quality patches decreased infected occupancy, while a shift towards high quality patches increased infected occupancy (Fig~\ref{sens}(b)).  

However, at high pathogen longevities, high quality patch distributions had a net positive effect on total occupancy relative to the other patch quality distributions (Fig~\ref{sens}(c)).  Due to their lower extinction rates, high quality patches are able to maintain stable occupancy despite widespread infection.  On the other hand, at high pathogen longevities, the positive effect of low qualtiy distributions on susceptible occupancy is overwhelmed, and low quality patches are unable to support infected occupancy, resulting in an increase in host extinction events.  Thus, the role of the patch quality distribution and the presence of high or low quality patches is generally greatest for low to intermediate pathogen longevity -- once longevity is high enough, the pathogen persists and spreads easily enough that the effects of individual patches are washed out.  Then the role of patch quality lies primarily in its effect on extinction rates, and whether infected hosts can persist.

\begin{figure}
\centering
\includegraphics[height=7cm]{highlow(full).pdf}
\caption{The effect of the presence of high and low quality patches on (a) mean susceptible occupancy, (b) mean infected occupancy, and (c) mean total occupancy, measured relative to  a low variance metapopulation.  Red solid lines show hiqh quality metapopulations, blue dashed show low quality, and black shows high variance.}
\label{sens}
\end{figure}

To understand the mechanisms behind these effects, we further examined within patch dynamics in high variance metapopulations that produced endemic dynamics.  Because of their lower extinction rate and higher recolonization rate, high quality patches support more consistent occupancy than low quality patches (Fig~\ref{simvis}).  However, as an epidemic progresses, high quality patches increasingly support infected occupants, while low quality patches support infecteds infrequently (Fig~\ref{simvis}).  In addition, while high quality patches do support susceptible occupancy, they experience a greater number of infection events (susceptible to infected transitions) than low quality patches (Fig~\ref{infections}).  This relationship becomes weaker as longevity increases, due to the fact that as longevity increases, there are fewer susceptibles remaining in the metapopulation to become infected.  

\begin{figure}
\centering
\includegraphics[height = 9cm]{simvis.png}
\caption{Results from a single simulation with variance 0.2 and longevity 80.  The top panel shows total occupancy of susceptible (blue) and infected (red) patches over time.  The bottom panel shows the state -- susceptible (blue), infected (red), or unoccupied (white) -- of individual patches through time.  Patches are stacked vertically with the lowest quality ($\sim$ 0.25) at the bottom and the highest quality ($\sim$ 1.75) at the top.}
\label{simvis}
\end{figure}


\begin{figure}
\centering
\includegraphics[height=7cm]{infevents(full).pdf}
\caption{The number of infection events ($S$ to $I$ transitions) for individual patches in endemic simulations.  Light gray lines show smoothed fit for single simulations, while the thick black line shows the smoothed fit over all simulations.  The thin black lines show the smoothed fits for longevities of 40 (top) and 140 (bottom).}
\label{infections}
\end{figure}

\section{Discussion}
\label{discussion} 

The positive relationship between pathogen spread and environmental longevity demonstrated in the theoretical model above is well documented in models of empirical systems.  \cite{Almberg2011} find that, in plausible simulations of chronic wasting disease in deer, increasing prion survival in the environment increases the resulting force of infection and peak prevalence, with the  importance of reservoir transmission increasing through time.  Similarly, Breban et al 2009 show that environmental transmission of avian influenza virus allows it to persist in situations where direct transmission alone is insufficient.  \emph{Other examples.}

Moreover, for a given longevity, these simulations suggest that high variance metapopulations are more prone to widespread pathogen spread than low variance metapopulations (Fig~\ref{poutcome}.  The mechanisms underlying this behavior can be understood by drawing parallels with models of pathogen spread through a contact network.  In this context, our metapopulation can be viewed a network with patches as nodes connected through immigration (with the strength of the connection measured by connectivity).  Many studies have shown that networks with high degree variance (where a node's degree is roughly equivalent to its total connectivity in our metapopulation model) are easily invaded by pathogens, which are then able to spread much more rapidly than in less heterogeneous networks (\cite{Pastor-Satorras2001}).

These simulations also offer insight into the interaction between environmental longevity and habitat heterogeneity, suggesting that the specific effect of longevity on pathogen spread is mediated by the variance of the patch quality distribution.  In particular, metapopulations with high patch quality variance require lower environmental longevities to experience pandemic dynamics (Fig~\ref{poutcome}).  In this way, the effects of environmental longevity are exacerbated, or amplified, by a high variance patch quality distribution, especially at intermediate longevities.  

This interaction between environmental longevity and the variance of the patch quality distribution is largely driven by the presence of high quality patches.  Since a patch's connectivity is determined, at least in part, by its quality, high quality patches function as highly connected hubs in the metapopulation.  Specifically, high quality patches both attract more colonists from other patches and produce more colonists that spread to other patches.  These two properties have important implications for pathogen spread in the metapopulation. Attracting colonists helps position high quality patches as the ecological "traps" hypothesized above, wherein susceptible hosts repeatedly colonize high quality patches and subsequently become infected (Fig~\ref{infections}).  This infection pressure results in part from direct transmission from infected immigrants attracted to the high quality patch, reflecting results from the contact network literature that show that highly connected nodes have higher infection risk (\cite{Christley2005}, \cite{Keeling2005}), but is further augemented by environmental transmission.  

This trap effect, coupled with the relatively low extinction rate on high quality patches helps to create a stable platform from which the pathogen can spread through the rest of the metapopulation.  Indeed, high quality patches are effectively metapopulation scale superspreaders (\cite{Lloyd-Smith2005}).  Once the pathogen infects high quality patches, the larger number of colonists produced allow it to spread to the rest of the metapopulation relatively easily.  This process helps to maintain infected occupancy throughout the metapopulation, which in turn feed back on high quality patches to ensure a steady stream of infected colonists that maintain the trap effect.  Thus, through these two interacting mechanisms -- the trap and superspreader effects -- the presence of high quality patches serves to significantly increase pathogen spread and infected occupancy in the metapopulation (Fig~\ref{sens}).  

Despite the dominating effects of high quality patches, low quality patches still play an important role in some cases by facilitating increased susceptible occupancy (Fig~\ref{sens}). Even though individual low quality patches are unable to support consistent occupancy (Fig~\ref{simvis}), at the metapopulation level, the presence of low quality patches increases overall susceptible occupancy relative to a lower variance metapopulation, an effect that is strongest at intermediate longevities (Fig~\ref{sens}).   This phenomenon represents the other side of the hub role played by high quality patches in that low quality patches are relatively weakly connected to the metapopulation and are therefore infrequently colonized by infected individuals.  Moreover, the instability of low quality patches (i.e. the high extinction rate), together with the increased mortality of infected populations, means that infected populations do not persist long on low quality patches.  Because low quality patches are so infrequently colonized, at low to intermediate longevities, the environmental reservoir left behind by these infected populations generally decays before it has the opportunity to infect new susceptible colonists.  As a result, low quality patches represent a dead-end for the pathogen, reducing the number of patches from which it can spread, and thereby reducing infected occupancy (Fig~\ref{sens}).  

However, the importance of the patch quality distribution in pathogen spread depends on the pathogen's longevity, with the strongest effects of patch quality observed for intermediate longevities.  For these longevities, the timescale of pathogen decay roughly matches the timescale of the colonization and extinction processes.  In contrast, for low longevities, when the pathogen reservoir decays much faster than colonizations and extinctions occur, environmental transmission has little effect on disease dynamics and the effect of patch quality is relatively weak.  Conversely, when the pathogen decays much slower than colonizations and extinctions occur, i.e. high longevity, the pathogen reservoir persists on nearly all patches and the effects of patch quality are washed out.  In these cases, the role of patch quality lies primarily in its effect on extinction rates.  Due to their lower extinction rates, high quality patches are able to maintain stable occupancy despite widespread (pandemic) infection, while low quality patches are unable to support infected occupancy, resulting in an increase in host extinction events at the highest longevities.

These findings suggest the presence of a trade-off between facilitating metapopulation stability and inhibiting disease spread, with a pathogen's environmental longevity controlling the roles of different quality patches with respect to this trade-off.  At low to intermediate longevities, the addition of high quality patches functions similarly to increasing movement in the model developed by \cite{Hess1996}, increasing pathogen spread and reducing overall occupancy (Fig~\ref{sens}c).  In these cases, rather than serving as sources in the metapopulation, high quality patches  become sinks, and more importantly, they become ecological traps, which can be even more detrimental to the overall population (\cite{Kristan2003}).  From a management perspective, it is critical to understand this shift, as previous theoretical work suggests that management should focus on patches where conditions are most favorable (\cite{Strasser2010}).  Generally, this means that maintaining high quality habitat should form the focus of managers, but Strasser et al. (2010) show that stochastic disturbance (e.g. disease) can lead to cases where focusing on low quality patches is more effective in increasing population growth rate.

However, as pathogen longevity increases, the trade-off shifts and the addition of high quality patches is a net benefit to overall occupancy, reflecting the findings of \cite{Gog2002}, who find that under background infection from an alternative host, increasing movement rarely reduces occupancy, despite facilitating pathogen spread.  \cite{Park2012} further develops this work by noting that the magnitude of the detrimental effects of increased moevement on occupancy depends on the relative strength of different transmission pathways.  Specifically, he finds that as environmental transmission becomes stronger (e.g. higher longevity) relative to direct transmission, then total occupancy increases monotonically with movement, i.e. movement is a net benefit to the metapopulation.  These studies, coupled with the results presented here, suggest that when there is a persistent source of infection throughout the metapopulation (e.g. when the pathogen is long-lived in the environment), factors that facilitate metapopulation stability (i.e. increased movement, high quality patches) are a greater benefit than factors that inhibit pathogen spread (i.e. decreased movement, low quality patches).  

Spatial structure and disease spread represent two major factors influencing the dynamics of wildlife populations.  As we have seen, in cases where the pathogen is able to persist and remain infectious in the environment, these two factors can interact in complex and interesting ways.  Failing to consider these interactions and their effect on the roles played by high and low quality habitat risks overlooking the importance of low-quality habitat for the persistence of wildlife populations and the importance of high-quality habitat for the persistence of wildlife disease.

\clearpage

\section{Appendix}

\subsection{Sensitivity Analyses}

\subsubsection{Lattice structure}

\begin{figure}[h!]
\centering
\includegraphics[height=5.5cm]{metapop(x0).pdf}
\caption{Percent of simulations resulting in (a) endemic disease (both susceptibles and infecteds persist) and (b) pandemic disease (no susceptibles persist), as a function of the pathogen's environmental longevity (half-life of infectivity decay) and the variance of the patch quality distribution.}
\label{poutcome_x0}
\end{figure}   

\begin{figure}
\centering
\includegraphics[height=7cm]{highlow(x0).pdf}
\caption{The effect of the presence of high and low quality patches on (a) mean susceptible occupancy, (b) mean infected occupancy, and (c) mean total occupancy, measured relative to  a low variance metapopulation.  Red solid lines show hiqh quality metapopulations, blue dashed show low quality, and black shows high variance.}
\label{sens_x0}
\end{figure}

\begin{figure}
\centering
\includegraphics[height=7cm]{infevents(x0).pdf}
\caption{The number of infection events ($S$ to $I$ transitions) for individual patches in endemic simulations.  Light gray lines show smoothed fit for single simulations, while the thick black line shows the smoothed fit over all simulations.  The thin black lines show the smoothed fits for longevities of 40 (top) and 140 (bottom).}
\label{infections_x0}
\end{figure}

\clearpage

\subsubsection{Extinction scaling, $\alpha = 0$}

\begin{figure}[h!]
\centering
\includegraphics[height=5.5cm]{metapop(x0).pdf}
\caption{Percent of simulations resulting in (a) endemic disease (both susceptibles and infecteds persist) and (b) pandemic disease (no susceptibles persist), as a function of the pathogen's environmental longevity (half-life of infectivity decay) and the variance of the patch quality distribution.}
\label{poutcome_x0}
\end{figure}   

\begin{figure}
\centering
\includegraphics[height=7cm]{highlow(x0).pdf}
\caption{The effect of the presence of high and low quality patches on (a) mean susceptible occupancy, (b) mean infected occupancy, and (c) mean total occupancy, measured relative to  a low variance metapopulation.  Red solid lines show hiqh quality metapopulations, blue dashed show low quality, and black shows high variance.}
\label{sens_x0}
\end{figure}

\begin{figure}
\centering
\includegraphics[height=7cm]{infevents(x0).pdf}
\caption{The number of infection events ($S$ to $I$ transitions) for individual patches in endemic simulations.  Light gray lines show smoothed fit for single simulations, while the thick black line shows the smoothed fit over all simulations.  The thin black lines show the smoothed fits for longevities of 40 (top) and 140 (bottom).}
\label{infections_x0}
\end{figure}

\clearpage

\subsubsection{Weak direct transmission, $\delta = 0.1$}

\begin{figure}[h!]
\centering
\includegraphics[height=5.5cm]{metapop(x0).pdf}
\caption{Percent of simulations resulting in (a) endemic disease (both susceptibles and infecteds persist) and (b) pandemic disease (no susceptibles persist), as a function of the pathogen's environmental longevity (half-life of infectivity decay) and the variance of the patch quality distribution.}
\label{poutcome_x0}
\end{figure}   

\begin{figure}
\centering
\includegraphics[height=7cm]{highlow(x0).pdf}
\caption{The effect of the presence of high and low quality patches on (a) mean susceptible occupancy, (b) mean infected occupancy, and (c) mean total occupancy, measured relative to  a low variance metapopulation.  Red solid lines show hiqh quality metapopulations, blue dashed show low quality, and black shows high variance.}
\label{sens_x0}
\end{figure}

\begin{figure}
\centering
\includegraphics[height=7cm]{infevents(x0).pdf}
\caption{The number of infection events ($S$ to $I$ transitions) for individual patches in endemic simulations.  Light gray lines show smoothed fit for single simulations, while the thick black line shows the smoothed fit over all simulations.  The thin black lines show the smoothed fits for longevities of 40 (top) and 140 (bottom).}
\label{infections_x0}
\end{figure}

\clearpage

\bibliographystyle{spbasic}     
\bibliography{metapop}   % name your BibTeX data base





\end{document}
