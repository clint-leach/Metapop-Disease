\documentclass{article} 

\usepackage{amsmath}
\usepackage{graphicx}
\usepackage{tabularx}
\usepackage{lineno}
\usepackage{hyperref}
\usepackage{color}
\usepackage{authblk}

\linenumbers

\begin{document}

\title{Modelling the creation of ecological traps by environmentally persistent pathogens}

\author[1]{Clinton B. Leach$^*$}
\author[1]{Colleen T. Webb}
\affil[1]{Department of Biology, Graduate Degree Program in Ecology, Colorado State University, Fort Collins, CO, USA}
\author[2]{Paul C. Cross}
\affil[2]{Northern Rocky Mountain Science Center, US Geological Survey, Bozeman, Montana, USA}
\affil[*]{Corresponding author: clint.leach@colostate.edu}


\maketitle


\begin{quote}
This draft manuscript is distributed solely for purposes of scientific peer review.  Its content is deliberative and pre decisional, so it must not be disclosed or released by reviewers.  Because the manuscript has not yet been approved for publication by the U.S. Geological Survey (USGS), it does not represent any official USGS finding or policy.
\end{quote}



\begin{abstract} 
Patterns of habitat quality are expected to play an important role in the dynamics of wildlife metapopulations, with the expectation that high quality habitat serves to stabilize and maintain occupancy.  However, we might expect that the costs and benefits of different quality habitat might be modulated, or possibly even reversed, by the presence of an environmentally transmitted pathogen.  We explore this possibility through the development of a stochastic patch occupancy model of host-pathogen dynamics, in which we can vary both the environmental longevity of the pathogen, and the distribution of habitat quality in the host metapopulation.
\end{abstract}

\section{Introduction}
\label{intro}

Many wildlife diseases are caused by pathogens that can persist, and remain infectious, for long periods of time in the environment.  Examples include chronic wasting disease (\cite{Miller2006}), anthrax (\cite{Dragon1995}), plague (\cite{Eisen2008}), and white nose syndrome (\cite{Lindner2011}), among others.  This environmental persistence creates environmental pathogen reservoirs from which hosts can become infected without direct contact with an infectious individual.  This additional transmission pathway can have important consequences for disease dynamics, with models showing that increased environmental persistence generally facilitates increased pathogen persistence and spread relative to direct transmission alone (\cite{Almberg2011}, \cite{Sharp2011}, \cite{Breban2009}). 

Since environmental transmission is spatially explicit (i.e. environmental reservoirs can only infect local residents), its role in disease dynamics may further depend on the spatial structure of the host population.  In particular, we expect that host population structure and movement will be influenced by heterogeneity in quality among the habitat patches in a metapopulation.  Indeed, the quality of a habitat patch can affect its extinction and colonization rates, as well as its contribution to the colonization of other empty patches (\cite{Moilanen1998}).  These processes in turn may then influence where environmental pathogen reservoirs get established and how they affect disease dynamics and host occupancy throughout the metapopulation.  

Specifically, we expect that high quality habitat patches, which support greater host density and traffic than lower quality habitat, might be more likely to form pathogen reservoirs.  As these reservoirs are undetectable to the host, we predict that high quality patches will continue to attract -- and infect -- susceptible immigrants, effectively creating an ecological trap (\cite{Almberg2011}).  In addition, the greater traffic through high quality patches may further facilitate pathogen spread by positioning high quality patches as metapopulation-scale superspreaders (\cite{Paull2012}).  Similarly, we expect that low quality patches, which see relatively less host traffic, will be less likely to develop pathogen reservoirs and thus may serve as refuges on which susceptible hosts can escape infection.  

Many, if not all, of the pathogens listed above affect spatially structured host populations (e.g. plague in prairie dog colonies, \cite{George2013}), and thus understanding how environmental transmission interacts with patterns of habitat quality is critical to managing disease in these systems.  In this study then, we seek to explore how a pathogen's environmental longevity (how long it can persist and remain infectious in the environment), and the distribution of habitat quality in a metapopulation interact to influence pathogen spread and host occupancy, with a specific interest in the roles played by high and low quality habitat.  

\section{Methods}
\label{methods}

\subsection{Model Structure}


To address the above questions, we developed a stochastic patch occupancy model (SPOM) in which each patch can be in one of three possible states: occupied by susceptible hosts (S), occupied by infectious hosts (I), and unoccupied by the host ($\emptyset$).  State transitions are governed by host colonization, extinction, and infection rates, where a susceptible population can become infectious either through direct contact with infectious immigrants, or through a local pathogen reservoir (Table~\ref{transitions}).

Building on the framework developed by \cite{Hanski2003}, the rate at which patch \emph{i}, with quality $A_i$, is colonized by individuals in state $X \in (S,I)$ is given by its connectivity:
\begin{equation}
C_{Xi}=A_i^{\xi_{im}} \sum_{i\neq j }\phi_jA_j^{\xi_{em}}\exp(-D d_{ij}),
\label{connectivity}
\end{equation}
where $\phi_j$ is an indicator function that is $1$ if patch $j$ is in state $X$ and is $0$ otherwise; $\xi_{im}$ and $\xi_{em}$ control how rates of immigration and emigration, respectively, scale with patch quality; $D$ is the inverse of mean dispersal distance; and $d_{ij}$ is the distance between patches $i$ and $j$.  Essentially, $C_{Xi}$ sums the colonization effort from all patches in state $X$ to patch $i$.  

The extinction rate of patch $i$ in state $X$ is given by:
\begin{equation}
E_{Xi}=\frac{\mu_X}{A_i^\alpha},
\end{equation}
where $\mu_X$ is the extinction rate of a patch of unit quality in state $X$ (here we assume that $\mu_I \geq \mu_S$ to account for disease-induced mortality), and $\alpha$ controls how extinction rate scales with patch quality.  When an infectious population goes extinct, we assume that the hosts leave behind an infectious pathogen reservoir on the patch.  

Susceptible populations are infected via direct contact with infectious colonists at rate $\delta \lambda_{Ii}$, where $\delta$ is the direct transmission probability.  In addition, environmental transmission can take place when a susceptible population occupies a previously infected patch, and occurs at rate:
\begin{equation}
\gamma(\tau_i)=\gamma_0exp(-r\tau_{i}),
\end{equation}
where $\gamma_0$ is the initial infection rate of the pathogen reservoir, $\tau_{i}$ is the time since last infectious occupancy ($\tau_{i} = \infty$, and thus $\gamma(\tau_i) = 0$, if the patch has never been occupied by infectious hosts), and $r$ is the pathogen's decay rate in the environment.

\subsection{Model Parameterization}

The model outlined above was parameterized in five different ways to explore the consequences of different assumptions (Table~\ref{params}).  The default model was parameterized  according to the following assumptions. (1) High quality patches maintain higher population sizes and therefore produce more colonists than low quality patches and have lower extinction rates than low quality patches ($\xi_{im}=0.5$, $\alpha=1$, \cite{Hanski2003}).  (2) High quality patches attract more colonists than low quality patches ($\xi_{em}=0.5$, \cite{Hanski2003}).  (3)  Connectivity among patches is determined entirely by quality ($d_{ij}=1$ for all $i \neq j$).  In addition to these assumptions, parameters were chosen so that without infection, approximately 0.75 of the metapopulation was occupied ($\mu_S = 0.1$, $D=5$), and a range of epidemiological behaviors -- disease-free, endemic disease, and pandemic disease (i.e. all host populations infectious) -- was feasible, depending on the pathogen's environmental longevity ($\mu_I = 0.5$, $\delta = 0.5$, $\gamma_0 = 0.5$). 

The baseline model above assumes that all patches are equally accessible from all other patches (i.e. are separated by distance 1), so to examine the effects of a more rigid spatial structure, we also implemented a model with patches arranged in a square lattice such that each patch is only accessible from its four neighboring patches (Local movement, Table~\ref{params}).  In addition, we explored a model with $\alpha = 0$ so that the extinction rate was constant with patch quality, simulating the effects of high environmental stochasticity, i.e. environmental stochasticity is strong enough that the population size of a patch does not affect its extinction rate (High env. stoc., Table~\ref{params}).  To explore the disease dynamics when transmission is driven largely by the environmental reservoir we parameterized a model with $\delta = 0.1$ to reduce the relative importance of direct transmission (Dominant reservoir trans., Table~\ref{params}).  Lastly, to simulate a more virulent pathogen, we parameterized a model with $\mu_I = 1$ to substantially increase disease-induced mortality (High disease mortality, Table~\ref{params}).

\subsection{Simulation studies}

For each of the above parameterizations, we performed two different simulation experiments to explore the combined effects of pathogen longevity and habitat quality on disease dynamics and metapopulation occupancy.  In the first of these, we investigated the effect of increasing habitat heterogeneity on the dynamics of pathogens with a range of environmental longevities.  For these simulations, we explored 10 values of pathogen longevity ranging from 0.1 to 3 times the expected residence time of a healthy population on a unit quality patch (i.e. 10 units of time), with values representing the half-life of the pathogen's infectivity and 10 values for the variance of the patch quality distribution, ranging from 0.02 to 0.2.  For each simulation (100 replicates for each of the 100 longevity-variance combinations), qualities for 100 patches were drawn from a uniform distribution with fixed mean 1 and given variance. In each case, an entirely susceptible population was simulated until it reached an approximate steady state, at which point a randomly chosen occupied patch was infected.  The state of the metapopulation was then tracked for 5000 time steps.  We recorded the proportion of patches in each state at the end of each simulation and classified the simulation as host extinction, disease-free (no infectious populations after 5000 time steps), endemic (both susceptible and infectious populations present), or pandemic (only infectious populations).  In addition, to identify the roles played by individual patches of varying quality, for every simulation, we recorded the quality of each patch, the proportion of time spent in each state (i.e. occupied by susceptible hosts, occupied by infectious hosts, and unoccupied), and the number of transitions between each state.  

The above simulations held the mean of the patch quality distribution constant and explored the effect of increasing habitat heterogeneity around that mean (i.e. by expanding the quality distribution symmetrically about 1).  To further parse out the relative influence of high and low quality patches separately, we performed a second series of simulations where we expanded the range of the uniform quality distribution to favor either high or low quality patches (i.e. by expanding the quality distribution asymmetrically about 1).  For these simulations, at each of the 10 longevities used above, we simulated 100 metapopulations for each of four patch quality distributions: low variance, (patch qualities ranging 0.75 to 1.25); intermediate variance, high quality (range 0.75 to 1.75); intermediate variance, low quality (range 0.25 to 1.25); and high variance (range 0.25 to 1.75). 

Continuous time stochastic simulations of the above model were implemented in the R language (\cite{R2014}) using the Gillespie algorithm (\cite{Gillespie1977}).  Code is available at \url{https://github.com/clint-leach/Metapop-Disease}.

\section{Results}
\label{results}

In the baseline parameterization, infectious occupancy increased substantially with longevity, while susceptible and total occupancy decreased (Supplemental).  However, increasing the variance of the patch quality distribution around a mean of 1 had relatively little effect on average occupancy (Supplemental).  The four alternative parameterizations (models with local movement, high environmental stochasticity, dominant reservoir transmission, and high disease-induced mortality) produced qualitatively similar results, though pathogen spread in these models was generally more limited (Supplemental).

We found that distributions with more low quality habitat (distribution ranging from 0.25 to 1.25) increased mean susceptible occupancy, while distributions with more high quality habitat (range from 0.75 to 1.75) decreased mean susceptible occupancy, with the largest differences observed at intermediate longevities (Fig~\ref{sens}(a)).  Conversely, low quality habitat distributions decreased infectious occupancy, while high quality distributions increased infectious occupancy (Fig~\ref{sens}(b)).  These general trends were consistent across the four other parameterizations (Supplemental).  

However, despite facilitating pathogen spread, at high pathogen longevities, high quality patch distributions had a net positive effect on total occupancy relative to low quality distributions (Fig~\ref{sens}(c)).  Due to their lower extinction rates and high connectivity, high quality patches were able to maintain stable occupancy despite the widespread infection facilitated by high pathogen longevity.  On the other hand, though low quality patch distributions inhibited pathogen spread at intermediate longevities, when high pathogen longevities led to pandemic dynamics, low quality distributions were unable to support infectious occupancy, resulting in an increase in host extinction events and a decrease in mean total occupancy.  

The other model parameterizations produced a similar qualitative pattern, with the differences between total occupancy on high and low quality patches generally producing a "u" shape as a function of pathogen longevity (Fig~\ref{highvlow}).  However, the five parameterizations differed in the magnitude and sign of this general relationship.  High environmental stochasticity (i.e. $\alpha = 0$) produced very similar results to the default model, though with a larger positive effect of low quality habitat at intermediate longevity (Fig~\ref{highvlow}(c)).  In simulations with reservoir transmission the dominant infection pathway ($\delta = 0.1$), the high quality patch distributions always produced occupancy higher than (or equal to ) the low quality patch distributions (Fig~\ref{highvlow}(d)), while in the models with local movement or high disease-induced mortality, low quality patch distributions actually produced higher occupancies than high quality distributions at high longevities (Fig~\ref{highvlow}(b, e)).

To understand the mechanisms behind these effects, we further examined within patch infection dynamics.  To compare across the full range of patch qualities, and to ensure the opportunity for both suscepible and infectious occupancy, we limit this analysis to high variance metapopulations (with qualities ranging from 0.25 to 1.75) that produced endemic dynamics.  Because of their lower extinction rate and higher recolonization rate, high quality patches supported more consistent occupancy than low quality patches (Fig~\ref{simvis}).  However, as the epidemic progressed, high quality patches increasingly supported infectious occupants, while low quality patches supported infectious hosts only briefly and infrequently (Fig~\ref{simvis}).  In addition, while high quality patches did support susceptible occupancy, they experienced a greater number of infection events (susceptible to infectious transitions) than low quality patches (Fig~\ref{infections}).  A similar trend was observed across the other parameterizations, though the observed relationship between patch quality and number of infection events was shallower for the lattice and weak direct infection models (Supplemental).


\section{Discussion}
\label{discussion} 

This work demonstrates that the distribution of habitat quality in a metapopulation can have substantial impacts on the dynamics of environmentally transmitted pathogens and the resulting patterns of host occupancy.  These impacts are driven largely by the contrasting effects of low and high quality habitat, and their relative balance in the metapopulation.  These contrasting effects derive from differences in colonization and extinction rates, which in turn drive how low and high quality patches interact with and modulate pathogen transmission, either augmenting it in the case of high quality patches, or dampening it in the case of low quality habitat.   

When a patch's connectivity is determined by its quality (Eqn~\ref{connectivity}), high quality patches function as highly connected hubs in the metapopulation.  Attracting colonists helps position high quality patches as infectious "traps", wherein susceptible hosts repeatedly colonize high quality patches and subsequently become infected (Fig~\ref{infections}).  This increased infection pressure results from both transmission from a persistent environmental reservoir and from direct transmission from infectious immigrants attracted to the high quality patch, reflecting results from the contact network literature that show that highly connected nodes have higher infection risk (\cite{Christley2005}, \cite{Keeling2005}).  

The relatively low extinction rate on high quality patches can help to create a stable platform from which the pathogen can spread through the rest of the metapopulation.  Indeed, high quality patches are effectively metapopulation scale superspreaders (\cite{Lloyd-Smith2005}, \cite{Paull2012}).  Once the pathogen infects high quality patches, the larger number of host colonists produced by these patches allow it to spread to the rest of the metapopulation relatively easily.  This process helps to maintain infectious occupancy throughout the metapopulation, which in turn feeds back on high quality patches to ensure a steady stream of infectious colonists that help maintain the environmental reservoir and the trap effect.  Thus, through these two interacting mechanisms -- the trap and superspreader effects -- the presence of high quality patches serves to significantly increase pathogen spread and infectious occupancy in the metapopulation (Fig~\ref{sens}).  

In contrast to high quality habitat, low quality patches help to limit pathogen spread and increase susceptible occupancy (Fig~\ref{sens}). Even though individual low quality patches, due to their high extinction rates and low colonization rates, are unable to support consistent occupancy (Fig~\ref{simvis}), the presence of low quality patches increases overall susceptible occupancy relative to a lower variance metapopulation (Fig~\ref{sens}).  This phenomenon represents the other side of the hub role played by high quality patches in that low quality patches are relatively weakly connected to the metapopulation and are therefore infrequently colonized by infectious individuals.  Moreover, the high extinction rate, together with the increased mortality of infectious populations, means that infectious populations do not persist long on low quality patches and the environmental reservoir left behind by these infectious populations may decay before it has the opportunity to infect new susceptible colonists.  As a result, low quality patches effectively represent a dead-end for the pathogen, reducing the number of patches from which it can spread, and thereby reducing infectious occupancy (Fig~\ref{sens}). 

Though there is a consistent trend in the effect of low and high quality habitat on pathogen spread, the net consequences of these effects for total occupancy vary depending on the structure and dynamics of the metapopulation.  Specifically, the findings that low quality habitat inhibits disease spread while high quality habitat enhances it suggest the presence of a trade-off between between processes that facilitate metapopulation stability at the expense of disease spread, and processes that hinder disease spread at the expense of metapopulation stability.  This work thus expands on previous studies of disease spread in a metapopulation that focused on movement rate as the driver of this trade-off, with increased movement both improving metapopulation stability and increasing disease spread (\cite{Hess1996}, \cite{Gog2002}, \cite{Park2012}).  Shifting the patch quality distribution towards high quality patches in our model is in many ways analogous to increasing movement in the models of Hess, Gog, and Park, but by including habitat heterogeneity and metapopulation structure, we are able to investigate the role of individual patches and their contribution to total occupancy.

These contributions, and the relative strength of the above trade-offs, depend largely on the role of connectivity, how it is affected by patch quality, and the resulting strength and structure of the different avenues of pathogen transmission.  Broadly, we see three different scenarios that roughly correspond to the three sections of the "u"-shaped relationships in Fig~\ref{highvlow}: conditions that support very weak pathogen spread, conditions that facilitate intermediate/endemic pathogen spread, and conditions that produce easy/pandemic pathogen spread.  When pathogen spread is weak (e.g. due to low longevity; the top left of the "u"), high quality patches are a net benefit to the metapopulation, as their connectivity and stability facilitate greater occupancy with little additional pathogen spread.  

However, in cases where the pathogen is able to spread more easily (i.e. at an endemic level; the trough of the "u"), the addition of high quality patches functions similarly to increasing movement in the model developed by \cite{Hess1996}: increasing pathogen spread and as a consequence reducing overall occupancy (Fig~\ref{sens}c).  In these cases, the reduction in occupancy from the effect of high quality habitat on pathogen spread outweighs the positive consequences of increased metapopulation connectivity and stability.  The shift in this trade-off can arise from conditions that change either side of that balance.  High quality habitat distributions require lower pathogen longevities for pathogen spread to begin to take off, relative to low quality habitat distributions.  Thus, for intermediate longevities, high quality distributions experience considerably greater pathogen spread and lower occupancy than low quality distributions.

The trade-off can also switch in cases that disrupt the positive benefits of high quality habitat for metapopulation stability, i.e. pathogens with high disease-induced mortality or when dispersal is limited by metapopulation structure.  In these cases, rather than serving as sources that increase metapopulation occupancy, high quality patches become sinks.  More importantly, they become ecological traps, which can be even more detrimental to the overall population (\cite{Kristan2003}).  From a management perspective, it is critical to understand this shift, as previous theoretical work suggests that management should focus on maintaining patches where conditions are most favorable (\cite{Strasser2010}).  Generally, this means that managers should focus on maintaining high quality habitat, but Strasser et al. (2010) show that stochastic disturbance (e.g. disease-induced mortality) can lead to cases where focusing on low quality patches is more effective in increasing population growth rate.  Indeed, in these situations, our model suggests that low quality patch distributions, by hindering the spread of the pathogen and providing refuges for susceptible hosts, can produce higher occupancy than high quality distributions.  

As pathogen longevity increases and environmental transmission becomes more important relative to direct transmission (i.e. the top right of the "u") the trade-off shifts back and the addition of high quality patches is again a net benefit to overall occupancy.  In these situations, environmental reservoirs overwhelm the susceptible refuges on low quality patches, such that they remain contaminated even across infrequent colonization events.  Under these conditions of widespread environmental transmission, high quality patches help to maintain metapopulation stability in the face of disease-induced mortality, thereby increasing total occupancy relative to more balanced or low quality patch distributions.  

These dynamics reflect the findings of \cite{Gog2002}, who found that under strong background infection from an alternative host, increasing movement increased total occupancy, despite facilitating pathogen spread.  \cite{Park2012}, by adding environmental transmission to the mix, found more generally that in cases where background and environmental transmission were strong relative to direct transmission, increasing movement increased total occupancy monotonically, i.e. movement was a net benefit to the metapopulation.  These studies, coupled with the results presented here, suggest that when there is a persistent source of infection throughout the metapopulation (e.g. when the pathogen is long-lived in the environment and direct infection is relatively less important), factors that facilitate metapopulation stability (i.e. increased movement, high quality patches) are a greater benefit than factors that inhibit pathogen spread (i.e. decreased movement, low quality patches).   

The different scenarios and parameterizations explored here represent a broad sample of the metapopulation dynamics that we might expect to observe in empirical systems.  The consistent "u"-shaped trend in the relative benefit of low and high quality habitat as a function of pathogen longevity suggest that these trade-offs between metapopulation stability and pathogen spread are likely to be fairly general.  By evaluating the relative importance of direct versus environmental transmission, along with the effect of habitat quality on connectivity and extinction, we can begin to diagnose where empirical systems (e.g. plague in prairie dogs, or white nose syndrome in bats) fall on this spectrum and the corresponding consequences for management.

Spatial structure and disease spread represent two major factors influencing the dynamics of wildlife populations.  As we have seen, in cases where the pathogen is able to persist and remain infectious in the environment, these two factors can interact in complex ways.  Failing to consider these interactions and their effect on the roles played by high and low quality habitat risks overlooking the importance of low-quality habitat for the persistence of wildlife populations and the importance of high-quality habitat for the persistence of wildlife disease.

\section{Acknowledgements}

CBL would like to thank members of the Webb Lab, past and present, for several rounds of useful feedback.  This work was supported by USGS Cooperative Agreement 101485, and NSF Graduate Research Fellowship, No. DGE-1321845.

Any mention of trade, product, or firm names is for descriptive purposes only and does not imply endorsement by the U.S. Government.

\clearpage

\bibliographystyle{vancouver}     
\bibliography{metapop}   % name your BibTeX data base

\clearpage

\begin{figure}
\caption{The effect of the presence of high and low quality patches on (a) median susceptible occupancy, (b) median infectious occupancy, and (c) median total occupancy.  Red lines show hiqh quality metapopulations (patch qualities roughly between 0.75 and 1.75), blue lines show low quality metapopulations (patch qualities roughly between 0.25 and 1.25).}
\label{sens}
\end{figure}

\begin{figure}
\centering
\caption{Differences between the median total occupancy of high quality patch distributions (quality between 0.75 and 1.75) and low quality patch distributions (quality between 0.25 to 1.25) across the five different model parameterizations. (a) default model, (b) local movement, (c) high environmental stochasticity, (d) dominant reservoir transmission, (e) high disease-induced mortality.}
\label{highvlow}
\end{figure}

\begin{figure}
\centering
\caption{Results from a single representative simulation with variance 0.2 and longevity of 0.3 times the expected residence time of a susceptible population on a unit quality patch.  The top panel shows total occupancy of susceptible (blue) and infectious (red) patches over time.  The bottom panel shows the state -- susceptible (blue), infectious (red), or unoccupied (white) -- of individual patches through time.  Patches are stacked vertically with the lowest quality ($\sim$ 0.25) at the bottom and the highest quality ($\sim$ 1.75) at the top.}
\label{simvis}
\end{figure}


\begin{figure}
\centering
\caption{The number of infection events ($S$ to $I$ transitions) for individual patches in endemic simulations.  Light gray lines show smoothed fit for single simulations, while the thick black line shows the smoothed trend over all simulations.}
\label{infections}
\end{figure}

\clearpage

\begin{table}
\caption{State transitions and their rates for patch $i$ in a metapopulation simulation.  $S$ denotes occupied by  susceptible hosts, $I$ denotes occupied infectious hosts, and $\emptyset$ denotes unoccupied by the host.  All patches are characterized by an environmental infection rate, $\gamma(\tau_i)$, where $\tau_i$ is the time since the patch was last occupied by infectious individuals.}
\begin{tabular}{llr}
State Transition & Process &  Rate \\
\hline
$S \rightarrow I$ & Infection (from contact or reservoir) & $\delta C_{Ii} + \gamma(\tau_i)$\\
$S \rightarrow \emptyset $ & Extinction (of susceptible) & $E_{Si}$\\
$I \rightarrow \emptyset $ & Extinction (of infectious) &  $E_{Ii}$ \\
$\emptyset \rightarrow S$ & Colonization (by susceptibles) & $C_{Si}$\\
$\emptyset \rightarrow I$ & Colonization (by infectious) & $C_{Ii}$\\
\label{transitions}
\end{tabular}
\end{table}

\begin{table}[h!]   
\caption{Parameters of the SPOM model, their meaning, and the values assigned under different parameterizations.  Empty cells indicate the same value as the default parameterization.}
\begin{tabular}{l p{4cm} p{1.5cm} p{1.5cm} p{1.5cm} p{1.5cm} p{1.5cm}}
Parameter & Interpretation &  Default & Local movement & High env. stoch. & Dominant reservoir trans. & High disease mortality \\
\hline
$\xi_{im}$ & Scaling parameter for effect of target patch quality on immigration & 0.5 & -- & -- & -- & --\\
$\xi_{em}$ & Scaling parameter for effect of target patch quality on emigration & 0.5 & -- & -- & --& --\\
$D$& Inverse of mean dispersal distance & 5 & 2 & -- & -- & --\\
$d_{ij}$ & Distance between patch $i$ and $j$ & 1 $\forall i \neq j$ & 1 $\forall i, j$ neighbors & -- & -- & --\\
\hline
$\mu_S$ & Extinction rate of unit quality susceptible patch & 0.1 & -- & -- & -- & --\\
$\mu_I$ & Extinction rate of unity quality infectious patch & 0.5 & -- & -- & -- & 1\\
$\alpha$ & Strength of environmental stochasticity & 1 & -- & 0 & -- & --\\
\hline
$\delta$ & Probability of direct infection & 0.5 & -- & -- & 0.1 & --\\
$\gamma_0$ & Initial rate of infection from reservoir patch & 0.5 & -- & -- & -- & --
\end{tabular}
\label{params}
\end{table}



\end{document}
